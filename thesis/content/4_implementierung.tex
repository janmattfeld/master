\chapter{Implementierung}

\section{Testumgebung: OpenStack, Docker \& Hyrise-R}



\section{Multi-Cloud-Bibliotheken}

Ziel ist die Implementierung eines externen Broker-Services oder die Aufwertung einer verteilten Anwendung für den automatischen Betrieb in mehreren Clouds. Da unabhängige Cloud-Provider keine einheitlichen APIs anbieten, stellt dieses Kapitel verschiedene Bibliotheken vor, um möglichst viel der zusätzlichen Komplexität zu verbergen.

Ohne weitere Bibliotheken müsste für jede zu berücksichtigende Cloud das jeweilige SDK eingebunden werden. Auch Namensgebung, Architektur und Prozesse unterscheiden sich von Anbieter zu Anbieter. 

Durch den Einsatz einer Dritt-Bibliothek ergibt sich allerdings eine potentielle Schwachstelle. Falls diese fehlerhaft ist, oder gar nicht weiter entwickelt wird, gefährdet dies das ganze Projekt. Historie und Zukunftschancen spielen bei der Auswahl eine zentrale Rolle. Im Optimalfall abstrahiert die Bibliothek Änderungen der Provider-SDKs. Ob und wie groß die Arbeitserleichterung ausfällt, prüft der Praxisteil.

Im Folgenden untersuchen wir die Eignung der populärsten Bibliotheken. Wichtigste Komponente ist dabei das Computing-Modul. Wünschenswert wäre auch Container-Unterstützung, um Images anbieterunabhängig bereitzustellen. Gestartete Anwendungskomponenten erfordern für die erste Erreichbarkeit oft Zugriff auf die DNS-Einstellungen der Cloud. Optional ist die Unterstützung von Content Delivery Networks, Speicher- und Backup-Diensten.


\begin{table*}\centering
	\begin{minipage}{\textwidth}
	\caption{Übersicht freier Multi-Cloud-Bibliotheken. Mit $*$ gekennzeichnete Eigenschaften sind experimentell. Aufgeführt sind nur die populärsten Cloud-Provider, die Bibliotheken können darüber hinaus weitere unterstützen.}
	\ra{1.3}
	\begin{tabularx}{\textwidth}{>{\centering}XXXX} \toprule
		Projekt & Cloud-Provider & Cloud-Services & Features\\ \midrule
		Apache Libcloud (Python)\footnotemark & OpenStack, AWS, Azure, GCP, Docker & Compute, Container, DNS, Load Balancer, Storage, Backup & Cost, Location\\
		Apache jclouds (Java)\footnotemark & OpenStack, AWS, Azure$*$, GCP, Docker & Compute, Container, Load Balancer$*$, Storage & Cost, Location\\
		PkgCloud (Node.js)\footnotemark & OpenStack, AWS, Azure & Compute, Load Balancer, Storage$*$, DNS$*$ & --\\
		Libretto (Go)\footnotemark & OpenStack, AWS, Azure, GCP & Compute & --\\
		Fog (Ruby)\footnotemark & OpenStack, AWS, GCP & Compute, DNS, Storage & Cost$*$\\
		\bottomrule
	\end{tabularx}
	\label{tab:bibliotheken}
	\vspace{100}
	\footnotetext[1]{\url{https://libcloud.apache.org/}}
	\footnotetext[2]{\url{https://jclouds.apache.org/}}
	\footnotetext[3]{\url{https://github.com/pkgcloud/pkgcloud/}}
	\footnotetext[4]{\url{https://github.com/apcera/libretto/}}
	\footnotetext[5]{\url{http://fog.io/}}
\end{minipage}  
\end{table*}

\autoref*{tab:bibliotheken} listet die untersuchten Bibliotheken mit unterstützten Cloud-Providern, Services und weiteren Features. Letzteres sind Zugriff auf Preisinformationen des Anbieters und Standortinformationen der Rechenzentren. Zusätzlich sollten die Projekte kontinuierlich weiterentwickelt werden, eine aktive Community besitzen und gut dokumentiert sein. Alle sind Open Source und unter einer freien Lizenz verfügbar.

\begin{description}
	
	\item[Apache jclouds] existiert schon seit 2009. Es unterstützt zumindest experimentell die wichtigsten Provider, aber nicht alle Services: DNS ist nicht vorhanden, Container-Unterstützung gibt es nur für Docker. Die Bibliothek ist gut getestet, dokumentiert, und mit zahlreichen Beispielen ausgestattet. Durch Java ist sie außerdem typsicher.

	\item[Apache Libcloud] vereint viele Vorteile: Es unterstützt neben OpenStack, als Referenz für Private-Cloud-Installationen, alle großen und kleinen Cloud-Provider mit allen Kernservices. Besonders interessant ist der Container-Support für Docker, Kubernetes, Amazon ECS und die Google Container Engine. Entsprechend gepackte Anwendungen könnten in einer Vielzahl von Clouds ohne weitere Änderungen ausgeführt werden.

	\item[Fog] integriert die wichtigsten Anbieter und Services. Die Community rund um Fog ist aktiv und die Bibliothek wird häufig eingesetzt. Besonders interessant sind die bereitgestellten Mocks, die Tests des neuen Services erleichtern sollen. Zumindest für OpenStack wird Metering unterstützt. Eine einheitliche Namensgebung der verschiedenen Cloud-Produkte existiert nicht.

	\item[Libretto] beschränkt sich ausdrücklich auf die Compute-Funktionalität mithilfe virtueller Maschinen. Das zugehörige Projekt ist aktiv, kommt aufgrund der fehlenden Funktionalität aber nicht infrage.

	\item[PkgCloud] ist die einzige bekannte Node.js-Bibliothek. Funktionsumfang und einheitliche Namensgebung der Cloud-Services sind überzeugend; leider wird die Bibliothek seit dem Verkauf des federführenden Unternehmens nicht mehr aktiv gepflegt. Bereits eingereichte Pull Requests werden nicht bearbeitet. Damit scheidet PkgCloud für das Projekt aus.

\end{description}

\noindent Libcloud fasst die verschiedenen Cloud-Angebote nicht nur in gemeinsamen Namensräumen zusammen, sondern normalisiert auch Leistungsklassen. Python erleichtert außerdem den Einstieg und fügt sich in viele Python-basierte Systemautomatisierungen ein. Diese Bibliothek wird also im weiteren Verlauf der Arbeit erprobt.

%young open source project, python, integrates SDKs instead of REST API, minimal set of features, unstable, OpenStack + AWS

\section{OpenStack Testbed}
http://trystack.org/ Facebook-Login