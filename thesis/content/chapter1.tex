\chapter{Die Cloud -- Chancen und Herausforderungen}

Dieses Kapitel definiert die grundlegenden Charakteristika eines Cloud-Dienstes, die verschiedenen Service-Ebenen, Liefermodelle, Akteure und ihre Verantwortlichkeiten. Aus diesen Definitionen entwickeln sich zwei grundlegende Herausforderungen der Cloud-Nutzung:

\begin{enumerate}
	\item Datenschutz/Vertraulichkeit
	\item Portabilität von Daten und Anwendungen
\end{enumerate}

% Unternehmens-Nutzers auf der PaaS-Ebene
\noindent Je nach Cloud-Nutzung ergeben sich hierfür verschiedene Lösungsansätze, die im weiteren Verlauf gegeneinander abgegrenzt werden.

\section{Eigenschaften eines Cloud-Dienstes}

Unabhängig von Liefer- und Servicemodell zeichnet sich ein Cloud-Dienst durch bestimmte Merkmale aus. Konkret definieren übereinstimmend \emph{NIST Cloud Computing Reference Architecture}, \emph{IETF} und \emph{BSI-Grundschutzkatalog} folgende Eigenschaften:
% http://ws680.nist.gov/publication/get_pdf.cfm?pub_id=909505 
% https://www.bsi.bund.de/DE/Themen/ITGrundschutz/ITGrundschutzKataloge/Inhalt/_content/m/m04/m04446.html?nn=6604968
% https://www.ietf.org/archive/id/draft-khasnabish-cloud-reference-framework-08.txt

\begin{description}

	\item[On-demand Self-service] Ressourcen werden vom Cloud-Kunden selbstständig über ein Portal oder eine Web-Schnittstelle angefordert und anschließend automatisch provisioniert.
	
	\item[Breitbandzugriff] Die gemieteten Ressourcen werden über ein Netzwerk, typischerweise das Internet, bereitgestellt. Der Zugriff erfolgt über Standard-Schnittstellen wie HTTP; kann also von überall erfolgen und ist im Regelfall nicht auf bestimmte Geräte oder Software beschränkt.
	
	\item[Geteilte Infrastruktur] Die zugrundeliegenden physikalischen Ressourcen werden virtualisiert und flexibel unter mehreren Kunden aufgeteilt. Die vorhandene Hardware wird so möglichst optimal ausgelastet. Gleichzeitig ergeben sich hierdurch Datenschutzbedenken; die Daten einzelner Mandaten müssen streng getrennt sein.
	
	\item[Elastizität] Durch einen hohen Grad an Automatisierung werden Ressourcen zeitnah zur Verfügung gestellt. Lastspitzen können so ohne manuelle Eingriffe abgefangen werden.

	\item[Messbarkeit] Die Ressourcennutzung ist messbar und wird kontinuierlich überwacht. Abgerechnet wird zum Beispiel nach CPU-Zeit, Speicherkapazität oder Anzahl genutzter IP-Adressen.
	
\end{description}

\noindent Von klassischem IT-Outsourcing grenzt es sich durch Self-Service, Skalierbarkeit und geteilte Infrastruktur ab. Diese Eigenschaften bieten Kunden theoretisch Flexibilität und Kostenvorteile. In der Lösungssuche sollen diese positiven Aspekte möglichst erhalten bleiben.


\section{Service-Ebenen}

Je nach Auswahl des Cloud-Angebots lassen sich verschiedene Kernebenen unterscheiden. Diese bauen jeweils aufeinander auf und verbergen die Komplexität der darunterliegenden Ebenen. Je weiter sich die Abstraktion von der physikalischen Ebene entfernt, desto weniger lässt sich das Angebot durch den Kunden anpassen:

\begin{description}
	
	\item[Infrastructure as a Service (IaaS)] Die klassische Bereitstellung von Infrastruktur wie virtuellen Maschinen, Speicherplatz und Netzwerkdienstleistungen. Der Kunde ist hier selbst für die Administration zuständig, muss also Einrichtung und Wartung von Betriebssystemen, Treibern und Middleware selbst verantworten.
	
	\item[Platform as a Service (PaaS)] Hier übernimmt der Cloud Provider die Bereitstellung der zuvor genannten Bestandteile. Der Kunde betreibt auf dieser Ebene eine selbst erstellte Anwendungssoftware. Über Bibliotheken und Schnittstellen des Cloud Providers greift er auf Laufzeitumgebungen, Datenbanken und Entwicklungswerkzeuge zu.
	
	\item[Function as a Service (FaaS)] Auch als \emph{Serverless Computing} populär: Entgegen des Namens arbeiten auch hier noch Server, diese sind für den Kunden jedoch weitestgehend unsichtbar. Es stellt eine Evolution des PaaS-Modells dar und ist besser als FaaS beschrieben -- der Kunde lädt nur noch Quellcode in die Cloud. Dieser wird nun Ereignis-getrieben ausgeführt, skaliert und abgerechnet. Im Gegensatz zu vielen PaaS-Angeboten fallen im Ruhebetrieb keine weiteren Kosten an.
	%https://www.crisp-research.com/serverless-infrastructure-der-schmale-grat-zwischen-einfachheit-und-kontrollverlust/
	
	\item[Software as a Service (SaaS)] Eine bestehende Anwendungssoftware wird komplett vom Cloud Provider bezogen. Die Verantwortlichkeit des Kunden beschränkt sich meist auf kleinere Anpassungen, Nutzerverwaltung und das Einspielen eigener Daten.
	
\end{description}

\noindent
Besonders interessant für eigene Entwicklungen im Rahmen des aktuellen Forschungskontextes sind dabei die Ebenen IaaS und PaaS. Sie bieten genug Flexibilität um die Fragestellung mit folgenden Produkten zu erproben:

\begin{enumerate}
	\item OpenStack als zentraler Infrastrukturprovider
	\item Die verteilte Forschungsdatenbank Hyrise-R als Beispielanwendung
\end{enumerate}

\noindent
Cloud Provider bieten darüber hinaus weitere Hilfs- und Verwaltungsdienste. Diese betreffen vor allem Konfiguration, Provisionierung, Monitoring und Abrechnung. Hierzu zählen aber auch Sicherheit, Vertraulichkeit und Portabilität. Diese drei Querschnittsthemen sollen im Zusammenhang mit den Dienstebenen weiter untersucht werden.

Die NIST-Klassifizierung unterscheidet hier speziell weitere, teils externe, Akteure wie Cloud-Auditoren und -Carrier. Dieser Einteilung folgt die Arbeit nicht. Stattdessen konzentriert sie sich auf die direkte Beziehung zwischen Cloud-Kunden und -Provider. Beide haben Risiken und Verantwortlichkeiten, die im nächsten Abschnitt besprochen werden.

\section{Risiken}

Moderne IT-Infrastruktur ist hochkomplex. Allein hierdurch ergibt sich ein großes Potential für Bedrohungen. Im Cloud Computing steigt das Risiko durch gemietete und geteilte Infrastruktur weiter. Zusätzlich zu allgemeinen Risiken sollen mögliche Auswirkungen auf folgende Eigenschaften beschrieben werden:

\begin{enumerate}
	\item Verfügbarkeit
	\item Vertraulichkeit
	\item Integrität
	\item Portabilität
\end{enumerate}

\noindent
Abhängig vom Einsatzzweck des geplanten Cloud-Services resultieren die Fragen: Welche Informationen und Prozesse müssen geschützt werden, welche Bedrohungen sind zu erwarten? Damit ist nicht nur Sicherheit gemeint, sondern alle Risiken, die den Erfolg eines Cloud-Projektes oder einer Organisation darüber hinaus bedrohen.

Dabei hilft, den möglichen Schaden im Voraus zu berechnen. Zu verarbeitende Daten sollten kategorisiert werden; dem BSI folgend sind diese vier Abstufungen denkbar:

\begin{enumerate}
	\item Privat- , Geschäfts- und Dienstgeheimnisse gemäß §§\,203 und 353\,b StGB
	\item Personenbezogene Daten gemäß §\,3 Absatz\,1 BDSG
	\item Verschlusssachen
	\item Sonstige Daten (weder Kategorie 1, noch 2, noch 3)	
\end{enumerate}

\noindent
Verschlusssachen der Kategorie drei meint hier alle Daten, deren Verlust, Veränderung oder unrechtmäßige Herausgabe sich nachteilig auswirken könnte. Die Abgrenzung der letzten beiden Kategorien erscheint oft schwierig, muss jedoch für jeden betriebenen Cloud-Service abgewogen werden.

Diese Arbeit konzentriert sich auf die Risiken, die direkt von Cloud-Kunden und -Providern auf den Ebenen IaaS und PaaS beeinflusst werden können. So ist zum Beispiel die Sicherheit der Client-Geräte, von den aus auf die Cloud-Dienste zugegriffen wird entscheidend, aber nicht Teil dieser Betrachtung.

Aus der Datenkategorie ergeben sich Anforderungen an die Risikoanalyse. Je höher und je wahrscheinlicher ein potentieller Schaden, desto aufwendiger und teurer sollte die Absicherung ausfallen. Grundsätzlich lassen sich zwei Kategorien von Risikofaktoren unterscheiden (Quelle):
% https://www.researchgate.net/publication/308691801_Taxonomy_for_Identification_of_Security_Issues_in_Cloud_Computing_Environments
\begin{itemize}
	\item Menschlich
	\item Technisch
\end{itemize}

\noindent Menschliche Fehlhandlungen sind entweder absichtlich oder unabsichtlich. Dies können Fehlinterpretation von SLAs, Manipulationen, Angriffe durch Social Engineering oder schlicht Inkompetenz sein. Im Cloud Computing treten diese Risiken verstärkt auf, da die Infrastruktur von Dritten betreut wird. Auch Gesetzesänderungen zählen zu diesen Risiken. Viele lassen sich durch passende Standard-Prozesse, Notfallpläne, Rechtemanagement und Audits vermindern.

Technisch gilt Ähnliches: Klassische Risiken wie der Ausfall von Hardware wird vom Cloud-Provider vor dem Kunden verborgen. Speziell auf Cloud-Projekte bezogen eröffnen sich aber auch neue Angriffsflächen wie die Cloud-Plattform selbst. Die Virtualisierungsebene kann durch mangelhafte Mandanten-Trennung Datenlecks öffnen. Viele Cloud-Provider arbeiten mit proprietären Protokollen, die Portabilität ist also eingeschränkt. Umso herausfordernder wird ein Notfallplan, der den Ausfall des Anbieters abfangen soll.

All diese Risiken müssen durch SLAs und Policys abgebildet werden. Diese sollten maschinenlesbar sein, um automatisiert angewandt und überprüft zu werden. Detaillierte Leitfäden hierzu bieten BSI, CSA. Insgesamt hängt das Risiko stark vom Bereitstellungsmodell ab, also von Standort und Nutzerkreis der Infrastruktur.
% Sichere Nutzung von Cloud-Diensten | Der sichere Weg in die Cloud
% IT-Grundschutz: M 2.535 Erstellung einer Sicherheitsrichtlinie für die Cloud-Nutzung
% BSI: Testierung nach BSI Anforderungskatalog Cloud Computing C5
% ISO/IEC 27001:2013, Information securitymanagement [13]
%Cloud Security Alliance (CSA) Security, Test & Assurance Registry (STAR): STAR Self Assess- ment, STAR Certification, STAR Attestation, C-STAR Assessment

%\section{Policy und SLA}

\section{Bereitstellungsmodelle und Multi-Cloud-Architekturen}

Cloud-Angeboten können von öffentlichen Anbietern oder intern bereitgestellt werden. Weiter klassifizieren lassen sich die Angebote nach Nutzerkreis, mit dem die Infrastruktur geteilt wird und Anzahl der genutzten Clouds:

\begin{description}
	
	\item[Public Cloud] Alle Leistungen werden von einem öffentlichem Anbieter bezogen. Dies sind zum Beispiel Amazon, Microsoft und Google. Die Infrastruktur wird unter mehreren Kunden flexibel aufgeteilt.
	
	\item[Private Cloud] Eine eigen- oder fremd-betriebene Infrastruktur mit exklusivem Zugriff für einen Kunden. Wird die Private Cloud im eigenen Datencenter betrieben, erhalten Kunden größtmögliche Vertraulichkeit. Gleichzeitig müssen aber Überkapazitäten vorgehalten werden, wodurch der Kostenvorteil kleiner als bei Nutzung öffentlicher Angebote ausfallen kann.
	
	\item[Hybird Cloud] Heterogene Infrastruktur mit Bestandteilen in privaten und öffentlichen Cloud-Umgebungen. Die öffentliche Cloud übernimmt hier oft die Speicherung großer Datenmengen und das Abfangen von Lastspitzen.
	
	\item[Community Cloud] Ein Zusammenschluss von Unternehmen, Behörden oder Forschungseinrichtungen, die gemeinsam eine Cloud-Infrastruktur betreiben. Die beteiligten Cloud-Anbieter bilden eine freiwillige \emph{Föderation} der gemeinsamen Ressourcen.
	
	\item[Multi-Cloud] Eine Erweiterung der Hybrid Cloud: Die Leistungen werden nicht nur aus einer privaten und einer öffentlichen Cloud bezogen, sondern explizit aus mehreren. Wichtiger Unterschied zur Föderation: Die Multi-Cloud-Umgebung wird vom Cloud-Anwender initiiert und verwaltet. Beteiligte Cloud-Provider wirken nicht aktiv mit und sind sich ihrer Partizipation meist nicht einmal bewusst.
	
\end{description}

\noindent
Von allen beschriebenen Bereitstellungsmodellen ist die Multi-Cloud am flexibelsten. Je nach technischen und nicht-funktionalen Anforderungen können Bestandteile der Cloud-Anwendungen in einer möglichst passenden Umgebung ausgeführt werden. Der Aufwand ist in diesem Modell allerdings auch am höchsten, denn der Auswahlprozess für einen Cloud-Provider muss für alle Anbieter einzeln durchlaufen werden.

Hierbei werden die Rahmenbedingungen geprüft, zum Beispiel Kosten, Standort der Rechenzentren und anwendbare Gesetzesgrundlage. Hinzu kommen technische Herausforderungen, da die Portabilität von Daten und Anwendungen zwischen verschiedenen Cloud-Providern oft eingeschränkt ist.

Fazit: Cloud-Nutzungsprozess wie in BSI "Sicher Nutzung von Cloud-Diensten" S.26\todo{Vergleichs-grafik zum neuen Prozess}, aber nicht Einweg, sondern kontinuierlich: Während des Betriebs kontinuierlich an neue Rahmenbedingungen anpassen, automatisch erkennen und migrieren. Daher: Cloud Broker.

Projektübergreifende Nutzung des Brokers. 
%\section{Cloud-Broker}

% Föderierte Cloud-Architekuren: Auch eine geteilte Nutzung zentraler Komponenten ist denkbar. In OpenStack teilen sich die verschiedenen Multi-Site-Clouds mindestens den Authentifizierungs-Service Keystone. Je nach Varainte (Cells, Regions, Availability Zones oder Host Aggregates) sind auch Dienste wie die Admin-Oberfläche oder Speicher nur einmal vorhanden. Diese Architektur spart Overhead, erfordert allerdings spezielle Anpassungen innerhalb der Cloud. Auch gehen einige Vorteile wie Ausfallsicherheit und Unabhängigkeit der zentralen Dienste wieder verloren. Eine Kombination mit weiteren Cloud-Providern ist denkbar, bleibt hier auf Grund der aufwändigen Einrichtung aber außen vor.



%\section{Related Work}
% 
%Eigene Rolle: Consumer. AGBs und SLAs meist nicht verhandelbar. Geteilte Verantwortung: Das beste daraus machen!
%
%Service-Taxonomy / Cloud-Typen
%
%
%Risiken
%
%
%Anforderungen
%
%Verwaltung über automatisierte Tools zur Orchestrierung.
%Kleine Marktübersicht?
%
%
%WAS IST EIN BROKER IN DIESEM KONTEXT?
%- Nicht Nutzerdaten werden geroutet.
%- Bestandteile verteilter Anwendungen.
%- Begründen.
%
%Klassifizierung der Broker.
%Ngrozev