\chapter{Hintergrund -- Chancen und Herausforderungen in der Cloud}

Dieses Kapitel definiert die grundlegenden Charakteristika eines Cloud-Dienstes, die verschiedenen Service-Ebenen, Liefermodelle, Akteure und ihre Verantwortlichkeiten. Aus diesen Definitionen entwickeln sich zwei grundlegende Herausforderungen der Cloud-Nutzung:

\begin{enumerate}
	\item Datenschutz/Vertraulichkeit
	\item Portabilität von Daten und Anwendungen
\end{enumerate}

% Unternehmens-Nutzers auf der PaaS-Ebene
\noindent Je nach Cloud-Nutzung ergeben sich hierfür verschiedene Lösungsansätze, die im weiteren Verlauf gegeneinander abgegrenzt werden.

\section{Eigenschaften eines Cloud-Dienstes}

Unabhängig von Liefer- und Servicemodell zeichnet sich ein Cloud-Dienst durch bestimmte Merkmale aus. Konkret definieren übereinstimmend \emph{NIST Cloud Computing Reference Architecture}, \emph{IETF} und \emph{BSI-Grundschutzkatalog} folgende Eigenschaften:
% http://ws680.nist.gov/publication/get_pdf.cfm?pub_id=909505 
% https://www.bsi.bund.de/DE/Themen/ITGrundschutz/ITGrundschutzKataloge/Inhalt/_content/m/m04/m04446.html?nn=6604968
% https://www.ietf.org/archive/id/draft-khasnabish-cloud-reference-framework-08.txt

\begin{description}

	\item[On-demand Self-service] Ressourcen werden vom Cloud-Kunden selbstständig über ein Portal oder eine Web-Schnittstelle angefordert und anschließend automatisch provisioniert.
	
	\item[Breitbandzugriff] Die gemieteten Ressourcen werden über ein Netzwerk, typischerweise das Internet, bereitgestellt. Der Zugriff erfolgt über Standard-Schnittstellen wie HTTP; kann also von überall erfolgen und ist im Regelfall nicht auf bestimmte Geräte oder Software beschränkt.
	
	\item[Geteilte Infrastruktur] Die zugrundeliegenden physikalischen Ressourcen werden virtualisiert und flexibel unter mehreren Kunden aufgeteilt. Die vorhandene Hardware wird so möglichst optimal ausgelastet. Gleichzeitig ergeben sich hierdurch Datenschutzbedenken; die Daten einzelner Mandaten müssen streng getrennt sein.
	
	\item[Elastizität] Durch einen hohen Grad an Automatisierung werden Ressourcen zeitnah zur Verfügung gestellt. Lastspitzen können ohne manuelle Eingriffe abgefangen werden.

	\item[Messbarkeit] Die Ressourcennutzung ist messbar und wird kontinuierlich überwacht. Abgerechnet wird zum Beispiel nach CPU-Zeit, Speicherkapazität oder Anzahl genutzter IP-Adressen.
	
\end{description}

\noindent Von klassischem IT-Outsourcing grenzt es sich durch Self-Service, Skalierbarkeit und geteilte Infrastruktur ab. Diese Eigenschaften bieten Kunden theoretisch Flexibilität und Kostenvorteile. In der Lösungssuche sollen diese positiven Aspekte möglichst erhalten bleiben.


\section{Service-Ebenen}

Je nach Auswahl des Cloud-Angebots lassen sich verschiedene Kernebenen unterscheiden. Diese bauen jeweils aufeinander auf und verbergen die Komplexität der darunterliegenden Ebenen. Je weiter sich die Abstraktion von der physikalischen Ebene entfernt, desto weniger lässt sich das Angebot durch den Kunden anpassen:

\begin{description}
	
	\item[Infrastructure as a Service (IaaS)] Die klassische Bereitstellung von Infrastruktur wie virtuellen Maschinen, Speicherplatz und Netzwerkdienstleistungen. Der Kunde ist hier selbst für die Administration zuständig, muss also Einrichtung und Wartung von Betriebssystemen, Treibern und Middleware selbst verantworten.
	
	\item[Container as a Service (CaaS)] Variante von IaaS, bei dem eine Laufzeitumgebung für Container bereitgestellt wird. In diesen sind alle Abhängigkeiten der Gastanwendung vorinstalliert und laufen auf dem bereits initialisierten Kernel des Hosts. Vorteil ist eine höhere Elastizität durch geringeren Overhead. Im Folgenden ist bei der Erwähnung von IaaS immer auch CaaS inkludiert.
	
	\item[Platform as a Service (PaaS)] Hier übernimmt der Cloud Provider die Bereitstellung der zuvor genannten Bestandteile. Der Kunde betreibt auf dieser Ebene eine selbst erstellte Anwendungssoftware. Über Bibliotheken und Schnittstellen des Cloud Providers greift er auf Laufzeitumgebungen, Datenbanken und Entwicklungswerkzeuge zu.
	
	\item[Function as a Service (FaaS)] Auch als \emph{Serverless Computing} populär: Entgegen des Namens arbeiten auch hier noch Server, diese sind für den Kunden jedoch weitestgehend unsichtbar. Es stellt eine Evolution des PaaS-Modells dar und ist besser als FaaS beschrieben -- der Kunde lädt nur noch Quellcode in die Cloud. Dieser wird nun Ereignis-getrieben ausgeführt, skaliert und abgerechnet. Im Gegensatz zu vielen PaaS-Angeboten fallen im Ruhebetrieb keine weiteren Kosten an.
	%https://www.crisp-research.com/serverless-infrastructure-der-schmale-grat-zwischen-einfachheit-und-kontrollverlust/
	
	\item[Software as a Service (SaaS)] Eine bestehende Anwendungssoftware wird komplett vom Cloud Provider bezogen. Die Verantwortlichkeit des Kunden beschränkt sich meist auf kleinere Anpassungen, Nutzerverwaltung und das Einspielen eigener Daten.
	
\end{description}

\noindent
Besonders interessant für eigene Entwicklungen im Rahmen des aktuellen Forschungskontextes sind dabei die Ebenen IaaS und CaaS. Sie bieten genug Flexibilität um die Fragestellung mit folgenden Produkten zu erproben:

\begin{enumerate}
	\item OpenStack als zentraler Infrastrukturprovider
	\item Die verteilte Forschungsdatenbank Hyrise-R als Beispielanwendung
\end{enumerate}

\noindent
Cloud Provider bieten darüber hinaus weitere Hilfs- und Verwaltungsdienste. Diese betreffen vor allem Konfiguration, Provisionierung, Monitoring und Abrechnung. Hierzu zählen aber auch Sicherheit, Vertraulichkeit und Portabilität. Diese drei Querschnittsthemen sollen im Zusammenhang mit den Dienstebenen weiter untersucht werden.

Die NIST-Klassifizierung unterscheidet hier speziell weitere, teils externe, Akteure wie Cloud-Auditoren und -Carrier. Dieser Einteilung folgt die Arbeit nicht. Stattdessen konzentriert sie sich auf die direkte Beziehung zwischen Cloud-Kunden und -Provider. Beide haben Risiken und Verantwortlichkeiten, die im nächsten Abschnitt besprochen werden.

\section{Risiken}

Moderne IT-Infrastruktur ist hochkomplex. Allein hierdurch ergibt sich ein großes Potential für Bedrohungen. Im Cloud Computing steigt das Risiko durch gemietete und geteilte Infrastruktur weiter. Zusätzlich zu allgemeinen Risiken sollen mögliche Auswirkungen auf folgende Eigenschaften beschrieben werden:

\begin{enumerate}
	\item Verfügbarkeit
	\item Vertraulichkeit
	\item Integrität
	\item Portabilität
\end{enumerate}

\noindent
Abhängig vom Einsatzzweck des geplanten Cloud-Services resultieren die Fragen: Welche Informationen und Prozesse müssen geschützt werden, welche Bedrohungen sind zu erwarten? Damit ist nicht nur Sicherheit gemeint, sondern alle Risiken, die den Erfolg eines Cloud-Projektes oder einer Organisation darüber hinaus bedrohen.

Möglicher Schaden muss im Voraus berechnet werden. Zu verarbeitende Daten sollten kategorisiert werden; dem BSI folgend sind diese vier Abstufungen denkbar:

\begin{enumerate}
	\item Privat- , Geschäfts- und Dienstgeheimnisse gemäß §§\,203 und 353\,b StGB
	\item Personenbezogene Daten gemäß §\,3 Absatz\,1 BDSG
	\item Verschlusssachen
	\item Sonstige Daten (weder Kategorie 1, noch 2, noch 3)	
\end{enumerate}

\noindent
Verschlusssachen der Kategorie drei meint hier alle Daten, deren Verlust, Veränderung oder unrechtmäßige Herausgabe sich nachteilig auswirken könnte. Die Abgrenzung der letzten beiden Kategorien erscheint oft schwierig, muss jedoch für jeden betriebenen Cloud-Service abgewogen werden.

Diese Arbeit konzentriert sich auf die Risiken, die direkt von Cloud-Kunden und -Providern auf den Ebenen IaaS und PaaS beeinflusst werden können. So ist zum Beispiel die Sicherheit der Client-Geräte, von denen auf die Cloud-Dienste zugegriffen wird entscheidend, aber nicht Teil dieser Betrachtung.

Aus der Datenkategorie ergeben sich Anforderungen an die Risikoanalyse. Je höher und je wahrscheinlicher ein potentieller Schaden, desto aufwendiger und teurer sollte die Absicherung ausfallen. Grundsätzlich lassen sich zwei Kategorien von Risikofaktoren unterscheiden (Quelle):
% https://www.researchgate.net/publication/308691801_Taxonomy_for_Identification_of_Security_Issues_in_Cloud_Computing_Environments
\begin{itemize}
	\item Menschlich
	\item Technisch
\end{itemize}

\noindent Menschliche Fehlhandlungen sind entweder absichtlich oder unabsichtlich. Dies können Fehlinterpretation von SLAs, Manipulationen, Angriffe durch Social Engineering oder schlicht Inkompetenz sein. Im Cloud Computing treten diese Risiken verstärkt auf, da die Infrastruktur von Dritten betreut wird. Auch Gesetzesänderungen zählen zu diesen Risiken. Viele lassen sich durch passende Standard-Prozesse, Notfallpläne, Rechtemanagement und Audits vermindern.

Technisch gilt Ähnliches: Klassische Risiken, wie der Ausfall von Hardware, wird vom Cloud-Provider vor dem Kunden verborgen. Speziell auf Cloud-Projekte bezogen eröffnen sich aber auch neue Angriffsflächen wie die Cloud-Plattform selbst. Die Virtualisierungsebene kann durch mangelhafte Mandanten-Trennung Datenlecks öffnen. Viele Cloud-Provider arbeiten mit proprietären Protokollen, die Portabilität ist also eingeschränkt. Umso herausfordernder wird ein Notfallplan, der den Ausfall des Anbieters abfangen soll.

All diese Risiken müssen durch SLAs und Policys abgebildet werden. Diese sollten maschinenlesbar sein, um automatisiert angewandt und überprüft zu werden. Detaillierte Leitfäden hierzu bieten BSI, CSA. Insgesamt hängt das Risiko stark vom Bereitstellungsmodell ab, also von Standort und Nutzerkreis der Infrastruktur.
% Sichere Nutzung von Cloud-Diensten | Der sichere Weg in die Cloud
% IT-Grundschutz: M 2.535 Erstellung einer Sicherheitsrichtlinie für die Cloud-Nutzung
% BSI: Testierung nach BSI Anforderungskatalog Cloud Computing C5
% ISO/IEC 27001:2013, Information securitymanagement [13]
%Cloud Security Alliance (CSA) Security, Test & Assurance Registry (STAR): STAR Self Assess- ment, STAR Certification, STAR Attestation, C-STAR Assessment

\section{Bereitstellungsmodelle und Multi-Cloud-Architekturen}

Cloud-Angeboten können von öffentlichen Anbietern oder intern bereitgestellt werden. Weiter differenzieren lassen sich die Angebote nach Nutzerkreis, mit dem die Infrastruktur geteilt wird und Anzahl der genutzten Clouds:

\begin{description}
	
	\item[Public Cloud] Alle Leistungen werden von einem öffentlichem Anbieter bezogen. Dies sind zum Beispiel Amazon, Microsoft und Google. Die Infrastruktur wird unter mehreren Kunden flexibel aufgeteilt.
	
	\item[Private Cloud] Eine eigen- oder fremd-betriebene Infrastruktur mit exklusivem Zugriff für einen Kunden. Wird die Private Cloud im eigenen Datencenter betrieben, erhalten Kunden größtmögliche Vertraulichkeit. Gleichzeitig müssen aber Überkapazitäten vorgehalten werden, wodurch der Kostenvorteil kleiner als bei Nutzung öffentlicher Angebote ausfallen kann.
	
	\item[Hybrid Cloud] Heterogene Infrastruktur mit Bestandteilen in privaten und öffentlichen Cloud-Umgebungen. Die öffentliche Cloud übernimmt hier oft die Speicherung großer Datenmengen und das Abfangen von Lastspitzen.
	
	\item[Community Cloud] Ein Zusammenschluss von Unternehmen, Behörden oder Forschungseinrichtungen, die gemeinsam eine Cloud-Infrastruktur betreiben. Die beteiligten Cloud-Anbieter bilden eine freiwillige \emph{Föderation} der gemeinsamen Ressourcen.
	
	\item[Multi-Cloud] Eine Erweiterung der Hybrid Cloud: Die Leistungen werden nicht nur aus einer privaten und einer öffentlichen Cloud bezogen, sondern explizit aus mehreren. Wichtiger Unterschied zur Föderation: Die Multi-Cloud-Umgebung wird vom Cloud-Anwender initiiert und verwaltet. Beteiligte Cloud-Provider wirken nicht aktiv mit und sind sich ihrer Partizipation meist nicht einmal bewusst.
	
\end{description}

\noindent
Von allen beschriebenen Bereitstellungsmodellen ist die Multi-Cloud am flexibelsten. Je nach technischen und nicht-funktionalen Anforderungen können Bestandteile der Cloud-Anwendungen in einer möglichst passenden Umgebung ausgeführt werden. Der Aufwand ist in diesem Modell allerdings auch am höchsten, denn der Auswahlprozess für einen Cloud-Provider muss für alle Anbieter einzeln durchlaufen werden.

Hierbei werden die Rahmenbedingungen geprüft; unter anderem Kosten, Standort der Rechenzentren und anwendbare Gesetzesgrundlage. Hinzu kommen technische Herausforderungen, da die Portabilität von Daten und Anwendungen zwischen verschiedenen Cloud-Providern oft eingeschränkt ist.

Um eine automatisierte Bewertung der Cloud-Angebote vorzunehmen, müssen die Anforderungen möglichst genau spezifiziert werden. Dazu dienen Policys und SLAs. Der folgende Abschnitt zeigt die wichtigsten Beispiele und erläutert ihre Bedeutung im Kontext der Arbeit.

%Fazit: Cloud-Nutzungsprozess wie in BSI "Sicher Nutzung von Cloud-Diensten" S.26\todo{Vergleichs-grafik zum neuen Prozess}, aber nicht Einweg, sondern kontinuierlich: Während des Betriebs kontinuierlich an neue Rahmenbedingungen anpassen, automatisch erkennen und migrieren. Daher: Cloud Broker.\todo{Fazit + Überleitung}
%
%Projektübergreifende Nutzung des Brokers. 

% ITIL

% Aufgaben einer CMP:
%https://www.gartner.com/doc/reprints?id=1-4KKGOTA&ct=171115&st=sb%3fsrc=so_5703fb3d92c20&cid=70134000001M5td
%

%\section{Related Work}
% 
%Eigene Rolle: Consumer. AGBs und SLAs meist nicht verhandelbar. Geteilte Verantwortung: Das beste daraus machen!
%
%Verwaltung über automatisierte Tools zur Orchestrierung.
%Kleine Marktübersicht?
%
%WAS IST EIN BROKER IN DIESEM KONTEXT?
%- Nicht Nutzerdaten werden geroutet.
%- Bestandteile verteilter Anwendungen.
%- Begründen.
%
%Klassifizierung der Broker.
%Ngrozev

%https://www.openservicebrokerapi.org
% Supported by all major corporations. On a PaaS-Level. Unified API for multi-cloud service consumption.
% Part of kubernetes Cloud Native Foundation (Linux Foundation) /https://www.cncf.io/ https://www.ietf.org/archive/id/draft-khasnabish-cloud-reference-framework-08.txt
% Eher theoretische Portabilität. Die Optimierung muss einzeln erfolgen. Keine Preisinformationen, oder STandort, oder ....
% The other two. Tatsächlich offene Spezifikation und in eigenen Projekten nutzbar.


% libcloud enthält aktuell keinen support für provider-interne orchestration werkzeuge (zb OpenStack HEAT). Aufgaben können also nicht weiter delegiert werden. Aufgabe des Frameworks.


\section{Policys, SLAs und Optimierungsgrößen}

Cloud-Nutzer und -Provider haben gegenseitige Erwartungen, die abgestimmt werden müssen. Innerhalb von Unternehmen gelten außerdem bestimmte Regeln zum Umgang mit Informationen (\emph{Compliance}), angelehnt an die Gesetzeslage und externe Zertifizierungen (ISO 27001). Über die gesetzlichen Regelungen hinaus kann ein Unternehmen weitere Ziele setzen: Kosteneinsparungen oder der Einsatz klimafreundlicher Energie. 

Dieser Abschnitt definiert die verschiedenen Formen für den späteren Einsatz im Multi-Cloud-Broker. Er liefert Verweise zu Standards und entwickelt Beispiele für den späteren Prototypen. Wir unterscheiden für die Verwendung im Broker:

\begin{description}
	\item[Policys] Maßnahmenorientiert, Regeln, \\
	z.\,B. \emph{Datenverarbeitung ausschließlich innerhalb der EU}
	\item[Service Level Agreements] Ergebnisorientiert, Vereinbarungen zur Dienstgüte, \\
	z.\,B. \emph{Verfügbarkeit über ein Jahr $99,99\,$\%}
	\item[Optimierungsgrößen] Gewichtete sonstige Ziele der Cloud-Nutzung, \\
	z.\,B. \emph{minimale Gesamtkosten bei Berücksichtigung aller Policys und SLAs}
\end{description}

\noindent Alle drei Themen beeinflussen sich gegenseitig: Richtlinien werden -- wenn möglich -- in das SLA übernommen, alle übrigen setzt der Cloud-Kunde selbst um. Eine Überschneidung ist ebenfalls möglich: neben einer 99,99\,\%-Uptime-Klausel kann es zusätzlich eine Replikations-Policy geben; beide bestimmen die Verfügbarkeit.

Typischerweise enthalten die Policys und SLAs weitere Informationen zu Vertragspartnern, Eskalationsmanagement, Vertragsstrafen und Laufzeit. Da diese Punkte für die Untersuchung nicht relevant sind, konzentrieren wir uns auf die in SLAs enthaltenen Metriken, sogenannte \emph{Service Level Objectives}. Wir nehmen zusätzlich einen vereinfachten Gültigkeitszeitraum über den gesamten Lebenszyklus eines Services an. 

Hierarchien innerhalb der Policys bleiben ebenso außen vor: So könnte es zum Beispiel die strategische Entscheidung geben, Daten nur innerhalb bestimmter Länder zu verarbeiten -- gleichzeitig gilt auf dem Service-Level eine weitere Einschränkung auf die EU. Wir konzentrieren uns direkt auf letzteren Fall. Weitere Ziele wie die Ausfallsicherheit unterscheiden sich ebenso individuell pro Service\todo{Definition: verteilte Anwendung}.

Sowohl Policys als auch SLAs sollten bestimmte Eigenschaften erfüllen. Eine Erweiterung der \emph{SMART-Kriterien} für Cloud-Computing könnte wie folgt aussehen:
%  G. T. Doran: There’s a S.M.A.R.T. way to write management’s goals and objectives. In: Management Review, 70. Jg., Nr. 11, 1981, S. 35–36.
\begin{enumerate}
	\item Erfüllbar
	\item Verständlich
	\item Nützlich
	\item Angemessen\,/\,Allgemein akzeptierbar
	\item Reproduzierbar
	\item Messbar
	\item Beeinflussbar
	\item Finanziell tragbar
\end{enumerate}

\noindent 
Einige Metriken sind typisch für Dienstleistungsverträge, zum Beispiel der Durchsatz. Andere wie die Löschfrist sind durch neuere Gesetzesvorgaben entstanden oder betreffen wie die Nachhaltigkeit das Image eines Unternehmens. Alle lassen sich grob in Kategorien sortieren; Allgemein, Performance, Zuverlässigkeit, Datenmanagement, Sicherheit, Datenschutz.
\todo{Tabelle mit Klassifizierung?}

Die folgende Auflistung zeigt die wichtigsten Policys und SLA-Metriken. Sie erläutert außerdem die Bedeutung für diese Arbeit:

\begin{description}
	
%	Klassifizierung
%Level: Unternehmen, Service (heir service, aber auch als Unternehmensrichtlinie denkbar)
%		Nutzer, Administration
%Decision- \& Enforcement-Point
% Typ: Hart, Weich, Optimierungsgröße
% Lebenszyklus des Services: Deployment, Betrieb, Terminierung

	\item[Redundanz] Wie oft soll ein bestimmter Dienst repliziert und parallel betrieben werden? Dieser Wert beeinflusst maßgeblich die Verfügbarkeit.

	
	\item[Geo-Lokation] Geographischer Standort des Datencenters. Mögliche Einschränkungen aufgrund von Datenschutzgesetzen. Beeinflusst außerdem die Antwortzeiten des Services.

	\item[Elastezität] Wie schnell können neue Ressourcen bereitgestellt oder wieder heruntergefahren werden? In einer IaaS-Umgebung ist dies meist der Zeitraum vom Start einer neuen VM bis zur Einsatzbereitschaft. Unsere Service-bezogene SLA misst zusätzlich die Zeit, bis eine neue Instanz tatsächlich Teil des Clusters ist, und einen Teil der Last abnimmt.
	
	\item[Agilität] Effizienzmetrik: Wie feingranular können die Ressourcen skaliert werden? Die Bedeutung für den Cloud-Broker ist vernachlässigbar:  Die verschiedenen Ressourcen-Angebote der Cloud Provider müssen im Voraus normalisiert werden. Zur Laufzeit werden sie nur noch verglichen.
	
	\item[Durchsatz] Wie viele Anfragen könne (pro Sekunde) bearbeitet und beantwortet werden? Die Anfrage ist vorher zu definieren. Typischerweise wird nicht das Mittel aller Anfragen gemessen, sondern von 99\% -- Ausreißer sind erlaubt.
	
	\item[Latenz] Zeit um eine Anfrage zu beantworten. Beeinflusst von Geo-Lokation, Datenbankstandort und Auslastung.
	
	\item[Datenstandort] Datenbank und Anwendung sollten immer in unmittelbarer Nähe zueinander platziert werden. Dies wirkt sich positiv auf die Latenz einer Anfrage aus, kann aber zu höheren Kosten führen.
	
	\item[Bevorzugen eigener Ressourcen] Solange in der eigenen Private Cloud Ressourcen vorhanden sind, sollten diese zuerst -- vor öffentlichen Drittangeboten -- genutzt werden. Auch hier müssen zusätzlich Leistungsmetriken wie die Latenz berücksichtigt werden.
	
	\item[Verfügbarkeitszeit (Uptime)] Zeitanteil, innerhalb dessen Anfragen den übrigen Metriken entsprechend beantwortet werden. Üblicherweise ab 99,9\,\% eines Jahres. Könnte durch die Gemeinsame Nutzung mehrerer Clouds erhöht werden.	
	
	\item[Wiederherstellungszeit] Durchschnittlicher Zeitraum, nachdem ein Service nach Ausfall wieder vollständig zur Verfügung steht. Dies beinhaltet die zügige und korrekte Beantwortung aller Anfragen sowie die Erfüllung aller übrigen SLOs.
	
	\item[Backup] Eine Sicherung von IaaS-Instanzen oder Datenbankinhalten in bestimmten Abständen, verteilt auf mehrere Speicherorte. In diesem Fall bezogen auf einen Service. Außerdem zu beachten: Die benötigte Zeit zur Wiederherstellung.
	
	\item[Zugriffskontrolle] Lesende und schreibende Aktionen auf Ressourcen müssen durch ein Identitäts- und Zugriffsmanagement geregelt und protokolliert werden. Dies ist jedoch ein eigenes Thema, das hier nicht weiter behandelt wird. Außerhalb der Administrationsebene implementiert die ausgeführte Anwendungssoftware oft eigene Mechanismen.
		
	\item[Benachrichtigungssystem] Ein Trigger-Action System: Bei Eintreten bestimmter Ereignisse wie dem Ausfall einer Systemkomponente soll automatisiert eine festgelegte Nachricht versendet werden, dies können eine E-Mail oder ein REST-Call sein.
	
	\item[Speichertypen] unterscheiden sich deutlich in Zugriffszeit, Transferrate, verfügbarer Kapazität, Zuverlässigkeit und Preis. IaaS-Angebote lassen oft die Wahl zwischen SSD und HDD. Auf PaaS-Ebene kann auch die Art des Datenbankservices gemeint sein: In-Memory, hybrid oder klassisch.
	
	\item[Löschfrist] Handels- und Steuerrecht verlangen für einige Daten die rechtssichere Speicherung und Revisionssicherheit. Umgekehrt gilt für personenbezogene Daten laut BDSG auch eine Löschpflicht, sobald die Daten nicht mehr für den ursprünglichen Zweck benötigt werden. Dies kann meist nur auf Datensatz-Ebene durchgesetzt werden. 
	
	Eine befristete Bereitstellung von Service-Instanzen ist jedoch auch denkbar. Die Löschung könnte dann im Rahmen von \emph{Aufräum}-Zyklen erfolgen und Kosten durch Überprovisionierung sparen.
	
	\item[Verschlüsselung] Je nach Service-Ebene entweder die Festplatte einer IaaS-Instanz oder der PaaS-Datenbankservice. Eine Verschlüsselung der Übertragungswege wird angenommen. Auf Datensatz- und Anwendungsebene sind eigene Implementierungen, unabhängig von der CMP üblich.
	
	\item[Kosten] Verbunden mit einem SLA bieten Cloud Provider Ressourcen zu einem bestimmten Preis. Dieser ist oft dynamisch, daher sollte eine Optimierung nach Prüfung aller anderen \emph{harten} Kriterien erfolgen.
	
	Den beteiligten Abteilungen hilft die Anzeige der Kosten pro Service. Die CMP sollte Auslastung und Preise dienstübergreifend optimieren.
	 	
	\item[Nachhaltigkeit/Energiequelle] Nutzt das Datencenter Ökostrom, wenn ja in welchem Umfang? Auch denkbar: Ausgleichsleistungen des Anbieters.
	
	%	https://www.researchgate.net/publication/234809221_Managing_energy_and_server_resources_in_hosting_centers 
	%	
	%	https://www.researchgate.net/publication/220831927_Energy-Aware_Server_Provisioning_and_Load_Dispatching_for_Connection-Intensive_Internet_Services 
	%	
	%	https://www.researchgate.net/publication/311880777_Power_efficient_server_consolidation_for_Cloud_data_center 
	
\end{description}

\noindent 
Diese Ziele gelten immer zusätzlich zu allgemeinen Hardware- und Softwareanforderungen. Voraussetzung für die Durchsetzung sind weitreichende Zusatzinformationen zum gegenwärtigen Zustand der Cloud und Umgebungsinformationen wie aktuellen Preisen. Innerhalb des \emph{Policy Enforcement Points}, also zum Beispiel der CMP, müssen diese Zusatzinformationen gesammelt werden.

Herausforderung hierbei: Die wichtigsten Cloud-Provider veröffentlichen die nötigen Informationen zu ihrem Angebot meist weder maschinenlesbar, noch in einem einheitlichen Format.
%TU Darmstadt: Scheint nicht erfolgreich zu sein. http://www.sla-ready.eu/sla-repository
Policys und besonders SLAs sind typischerweise in natürlicher Sprache verfasst. Für den Einsatz in einem Broker sollten sie jedoch streng formalisiert und damit maschinenlesbar sein, um automatisiert angewandt und überprüft zu werden.

Das folgende \autoref{cha:broker} zeigt daher -- neben bisherigen Broker-Forschungsarbeiten -- bestehende Schemata zur Darstellung von SLAs. Die weitere Besprechung der Policy-Gestaltung erfolgt im Implementierungsteil in \autoref{cha:implementierung}.