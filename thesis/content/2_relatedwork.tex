\chapter{Multi-Cloud-Anwendungs-Broker}

Um eine Anwendung automatisiert auf verschiedene Clouds zu verteilen ist ein Broker-Mechanismus nötig. Dieser sollte nach verschiedenen, festzulegenden Kriterien vorgehen. Konkret erfüllt ein Broker in der Regel folgende Aufgaben:

\begin{enumerate}
	\item Bereitstellen der Ressourcen für eine Applikation; starten einer virtuellen Maschine oder Reservierung von Speicherplatz
	\item Starten der Anwendung auf den vorher reservierten Ressourcen
	\item Verteilen eingehender Anfragen auf die reservierten Ressourcen
	\item Management der Ressourcen
\end{enumerate}

\noindent In einer Community Cloud ist denkbar, dass die Cloud-Provider selbst einen Mechanismus zum Brokering oder zumindest offene APIs bereitstellen. Der Broker wäre also Teil der Cloud. Möglich sind entweder ein zentraler Broker, der direkt auf Cloud-Interna zugreift, aber auch ein Peer-to-Peer-Verbund.

Für Multi-Clouds kommen diese Lösungen nicht infrage: sie bestehen aus mehreren unabhängigen, meist privaten, Cloud-Providern. Aufgrund gegenläufiger Geschäftsinteressen sind diese nicht an einer Föderation mit anderen Anbietern interessiert. Sie werden also weder Interna ihrer Cloud-Plattform anpassen, noch einheitliche APIs bereitstellen.

Stattdessen muss der Broker in einer Multi-Cloud-Umgebung extern bereitgestellt werden. In diesem Fall kann er entweder als eigenständiger Service in Form einer \emph{Cloud Management Platform} angelegt sein, oder von der verteilten Anwendung selbst implementiert werden. Selbst entwickelte CMPs oder integrierte Broker setzen oft auf Multi-Cloud-Bibliotheken wie Apache libcloud. Kapitel \todo{Link} bietet dazu einen Vergleich. Der folgende Abschnitt gibt eine Übersicht der bisherigen Forschung zu Inter- und Multi-Cloud-Brokern.

\section{Related Work}

Die vorgestellten Lösungen unterschieden sich in Architektur, Flexibilität und Funktionsumfang. Die vier Basiseigenschaften werden nicht von allen Arbeiten in vollem Umfang erfüllt. Zusätzlich denkbar ist das automatische Re-provisionieren anhand von 

\begin{enumerate}
	\item Lastspitzen oder Ausfällen von Hardware und Netzwerkressourcen (Monitoring)
	\item Geänderten Umfeldparametern wie der Gesetzgebung, Preise oder AGBs
	\item Nutzeränderungen
	\item Vorherigen Broker-Aktionen
\end{enumerate}


\section{Architektur}
% Komponenten

\section{Schemata: Cloud-Angebote, SLAs, Services}

\section{Matching}
%
%Angebot der Cloud Provider (SLA)
%Anforderungen
%
%Spezifizierung
%Schnittstellen
%
%Algorithmen