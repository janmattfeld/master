\chapter{Einleitung}

% Im Zentrum der Einleitung stehen die Vorstellung und Motivation des Themas der Arbeit und die genaue Auflistung der Fragestellungen (Wieso ist das Thema relevant?). Ebenso sollten schon einzelne Aspekte des Problems herausgearbeitet werden. Dabei ist es hilfreich, die zentralen Fragen aufzulisten, die im Rahmen der Arbeit beantwortet werden sollen. Außerdem sollte ein knapper Überblick gegeben werden, in welchen Schritten die Problembehandlung erfolgt: Hinführung zum Thema, Herleitung und Ausformulierung der Fragestellung, Abgrenzung des Themas (Angabe von Aspekten, die zum Thema gehören, aber ausgeklammert werden) und Aufbau der Arbeit (Begründung der Gliederung).

Cloud-Angebote sind allgegenwärtig und werden mittlerweile von der Mehrzahl deutscher Unternehmen genutzt. Laut Statista setzten bis 2016 mindestens fünfundsechzig Prozent aller Betriebe entsprechende Lösungen ein. 
%https://de.statista.com/statistik/daten/studie/177484/umfrage/einsatz-von-cloud-computing-in-deutschen-unternehmen-2011/
Besonders gefragt sind Infrastrukturdienste, wie Rechenleistung und Speicher. Gleich darauf folgen Softwareangebote und E-Mail-Hosting. Etwas abgeschlagen bleiben Plattformdienste wie Datenbanken und Ausführungsumgebungen. %https://de.statista.com/statistik/daten/studie/381830/umfrage/einsatzzwecke-von-cloud-computing-in-unternehmen-in-deutschland/
Der Trend zur Cloud wird sich vermutlich fortsetzen: So vervierfacht sich das prognostizierte Marktvolumen mit Cloud-Services bis 2020 auf über sechzehn Milliarden Euro allein im deutschen B2B-Markt. %
%https://de.statista.com/statistik/daten/studie/165458/umfrage/prognostiziertes-marktvolumen-fuer-cloud-computing-in-deutschland/

Die Cloud-Angebote sind für Unternehmen aller Größen attraktiv: Die Anschaffung eigener Infrastruktur entfällt, genauso wie deren Wartung durch eigenes Personal. Stattdessen lassen sich Ressourcen und Anwendungen einfach per Self-Service buchen und sind anschließend über das Internet von überall erreichbar. Der Umfang der gebuchten Leistungen lässt sich meist frei skalieren. Da die Angebote oft in kleinem Takt und verbrauchsgenau abgerechnet werden, ergeben sich so theoretisch Vorteile bei Flexibilität und Kosteneffizienz.

Demgegenüber stehen Vorbehalte bezüglich Datenschutz, denn die gemietete Infrastruktur teilen sich mehrere Kunden. Zugleich fordert aktuelle Gesetzgebung wie die Datenschutz-Grundverordnung unter anderem die Verarbeitung von personenbezogenen Daten europäischer Kunden ausschließlich innerhalb der EU. %http://data.consilium.europa.eu/doc/document/ST-12399-2016-INIT/en/pdf %Anbieter sind in der Pflicht. Haftung auch Gegenüber Endkunden, mit denen eigentlich keine direkte Geschäftsbeziehung besteht. Löschpflicht. Wo sind die Daten physikalisch? Portabilität Cloud->Kund und Cloud->Cloud
%Weitere Anforderungen Diese Regelungen sind keineswegs auf Europa begrenzt. Gleiches in der USA (Amazon Gov CLoud)
Gut neunzig Prozent aller Unternehmen achteten dementsprechend bei der Auswahl eines Cloud-Providers auf Rechts- und Server-Standorte in Deutschland. Für gut fünfzig Prozent außerdem unabdingbar: Hybrid-Cloud-Lösungen und individuelle SLAs.
%https://de.statista.com/statistik/daten/studie/545924/umfrage/kriterien-bei-der-auswahl-eines-cloud-providers-in-deutschen-unternehmen/

Portabilität? Weiter nicht-funktionale Anforderungen
Privat und geschäftlich führt kaum ein Weg an Google, Amazon und Microsoft vorbei. 


Dies lässt jedoch einige der weltweit größten Cloud-Anbieter außen vor. Amazon und Microsoft teilen seit 2016 über fünfzig Prozent des weltweiten Umsatzes mit Infrastrukturdiensten. %https://de.statista.com/statistik/daten/studie/754647/umfrage/marktanteile-am-umsatz-mit-infrastructure-as-a-service-weltweit/

%https://aws.amazon.com/security/security-bulletins/AWS-2018-013/ Intel-Prozessor-Bug, durch den geschützte Speicherbereiche angreifbar werden. Nach aktuellem Stand existierte diese Lücke mehr als ein halbes Jahr in fast jedem x86-System, und damit in fast jeder Cloud. Eine sichere Mandantentrennung war also nicht mehr gewährleistet.

%Amazon S3 Datenlecks

Lösung: Immer mehr Anwendungen laufen vollständig oder teilweise (hybrid) in der Cloud. Wie können die Vorteile der Cloudservices mit aktuellem Datenschutz zusammengebracht werden? Die Verwaltung der verteilten Applikationen sollte weiterhin automatische (Cloud Native) erfolgen.

Zeigen: 
- Eine Übersicht aktueller Cloudangebote und Datenschutzanforderungen.
- Related work: Gegenüberstellung Verschiedener kommerzieller und akademischer Projekte mit ähnlicher Zielsetzung. Fehlende Eigenschaften

Hyrise-R und OpenStack als Grundlage im Rahmen von SSCICLOPS

Organisationen und Unternehmen als Cloud-Nutzer emanzipieren. Da SLAs und weitere Rahmenbedingungen oft nicht verhandelbar sind, muss die Optimierung der Cloud-Nutzung in Eigenregie erfolgen.

Verantwortung übernehmen: Große Datenlecks aus ungesicherten Amazon-S3-Speichern.

- Vorschlag eines eigens entwickelten Multi-Cloud-Brokers, der auch nicht cloud-native Anwendungen über verschiedene Cloud-Provider verteilt. Dabei beachtet er SLAs und Datenschutzanforderungen.
- Technische Betrachtung des Prototypen.
- Future Work und Bewertung des Prototypen

Wie nützlich ist Apache libcloud (oder generell eine Multi-Cloud Library? Vergleich!)