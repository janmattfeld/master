% indem sie beispielsweise aus laufenden oder abgeschlossenen wissenschaftlichen Arbeiten bekannte Lösungen vertieft, verbreitert, bestätigt oder widerlegt oder aus einem neuen Blickwinkel betrachtet bzw. an einen neuen Anwendungsfall anpasst.

%Im Grundlagen-Teil sollen zentrale Begriffe definiert und eingeordnet werden. Es geht dabei nicht darum, Definitionen aus Lexika zu suchen; stattdessen sollen problemorientierte Definitionen gegeben werden. Häufig können einzelne Begriffe unterschiedlich weit oder eng definiert werden, sodass auch eine Diskussion unterschiedlicher Definitionsansätze hilfreich sein kann, bevor eine für die weitere Arbeit verbindliche Definition gewählt wird.
%Zudem sollte ein Überblick über die in der Literatur vorhandenen Methoden bzw. Lösungsansätze, der aktuelle Stand der Technik und verwandte Arbeiten gegeben werden. 

%Auch die einzelnen Abschnitte sollten Einleitung, Hauptteil, Schluss enthalten. Die Einleitung und Schluss sollten dabei die Überleitung zu den umschließenden Abschnitten realisieren.



