\section{DevStack virtualisiert inkl. Container-Support}

Während der Installation nimmt DevStack tief greifende Veränderungen am Hostsystem vor. Es müsste also auf einem separaten Server installiert werden. Dieser Abschnitt beschreibt den Versuch einer virtualisierten, reproduzierbaren DevStack-Testinstallation. Außerdem soll \emph{Zun} integriert werden. 

Ziel ist DevStack in einem Container auszuführen, genauso wie die darin gestarteten Compute-Nodes ebenso in einem -- nun verschachtelten -- Container bereitzustellen. Die Gründe hierfür sind zusammengefasst:

\begin{enumerate}
	\item Keine oder minimale Änderungen am Hostsystem
	\item Reproduzierbarer Testaufbau
	\item Schneller und rückstandsloser Reset
	\item Zustände (\emph{Snapshots}) speicherbar
	\item Schnelle Ausführung von Gastapplikationen
\end{enumerate}

\noindent Auch eine virtuelle Maschine löst die oben genannten Probleme. Theoretisch. Problematisch wird die Ausführungsgeschwindigkeit von Gastanwendungen in einem mit \emph{VirtualBox} virtualisierten OpenStack. Eine Lösung ist \emph{verschachteltes KVM}, das bereits in der Arbeit [1] erprobt wurde. Die Autoren empfehlen ihren Vorschlag bei bestehenden Erfahrungen mit \emph{libvirt}. Der damalige Versuchsaufbau stellt sich allerdings als instabil und nicht mehr reproduzierbar heraus.

\begin{figure}[ht]
	\centering
	\begin{subfigure}[b]{0.29\textwidth}
		\def\svgwidth{\linewidth}
		{\tiny \textsf{
			\includesvg{images/devstack-bare-metal}}}
		\caption{Bare-Metal-Installation}
		\label{fig:sub:devstack-bare-metal}
	\end{subfigure}\hfill%
	\begin{subfigure}[b]{0.31\textwidth}
		\def\svgwidth{\linewidth}
		{\tiny \textsf{
			\includesvg{images/devstack-vm}}}
		\caption{All-in-One-VM}
		\label{fig:sub:devstack-vm}
	\end{subfigure}\hfill%
	\begin{subfigure}[b]{0.31\textwidth}
		\def\svgwidth{\linewidth}
		{\tiny \textsf{
			\includesvg{images/devstack-docker}}}
		\caption{DevStack in Docker}
		\label{fig:sub:devstack-docker}
	\end{subfigure}
	
	\caption{Verschiedene Installationsvarianten für eine OpenStack-Testinstallation mit DevStack auf einem einzelnen Host -- inklusive Unterstützung für Docker-Compute-Container und klassische VMs. Eine direkte Installation verändert unwiderruflich das gesamte Host-System \emph{(a)}. Eine VM benötigt mehr Ressourcen und kann die Geschwindigkeit der Gastanwendungen negativ beeinflussen \emph{(b)}. Die Installation in einem Container schafft Abstraktion und Reproduzierbarkeit ohne Geschwindigkeitskompromisse. Die Gastcontainer nutzen weiterhin den Kernel des Host-OS \emph{(c)}.}
	\label{fig:devstack}
\end{figure}

\emph{LXD}-Container könnten sich ebenfalls eignen. Im Gegensatz zu Docker führen sie mehrere Prozesse aus, erinnern also mehr an eine klassische virtuelle Maschine (ohne deren Overhead). Laut Entwickler \emph{Canonical} fokussiert sich \emph{LXD} speziell auf IaaS-Aufgaben\footnote{\url{https://www.ubuntu.com/containers/lxd}}. Ein LXD-DevStack-Setup birgt allerdings die gleichen Hürden\footnote{\url{https://docs.openstack.org/devstack/latest/guides/lxc.html}} wie ein Docker-Setup \cite{graber:2016:openstack-lxd}. Beachtenswert ist noch das OpenStack-Projekt \emph{Kolla}, das jeden OpenStack-Dienst in einem eigenen Docker-Container installiert\footnote{\url{https://cloudbase.it/openstack-kolla-hyper-v/}}.

Um Container innerhalb von OpenStack auszuführen, gibt es mehrere, teils konkurrierende Projekte. Alle lassen sich über Plugins in DevStack einbinden. Dies sind die wichtigsten \cite{singh:2017:containers-openstack}:

\begin{description}
	
	\item[Zun\footnotemark]\footnotetext{\url{https://wiki.openstack.org/wiki/Zun}} Eigenständige OpenStack-API zum Starten und Verwalten von diversen Containertypen, inklusive \emph{Docker}.
	
	\item[Nova Docker\footnotemark]\footnotetext{\url{https://wiki.openstack.org/wiki/Docker}} Im Gegensatz zu \emph{Zun} erfolgt die Docker-Containerverwaltung über die bekannte Nova-API. Das Projekt wurde eingestellt.
	
	\item[Nova LXD\footnotemark]\footnotetext{\url{https://linuxcontainers.org/lxd/getting-started-openstack/}} Parallel zu \emph{Nova Docker} erfolgt der Zugriff über die Nova-API. Das Projekt wird von \emph{Canonical} aktiv vorangetrieben. Weiterer Teil ist die Automatisierung via \emph{Juju}.
	
	\item[Magnum\footnotemark]\footnotetext{\url{https://wiki.openstack.org/wiki/Magnum}} Eine Self-Service-Lösung zur Orchestrierung auf Basis von \emph{Heat}. Stellt automatisiert Container Orchestration Engines (COEs) wie \emph{Docker Swarm} und \emph{Kubernetes} bereit.
	
\end{description}

\noindent DevStack in Docker wurde bereits vor einiger Zeit umgesetzt\footnote{\url{https://github.com/ewindisch/dockenstack}}. Da das Projekt nicht mehr gepflegt wird und auf das ebenfalls beendete \emph{Nova Docker} aufsetzt, erfolgt die Neuimplementierung mit folgenden Änderungen:

\begin{itemize}
	
	\item Ubuntu-LTS-Basis-Image 14.04 $\Rightarrow$ 17.10
	\item Mehrprozessunterstützung per \emph{systemd}\footnote{\url{https://docs.openstack.org/devstack/latest/systemd.html}}
	\item OpenStack-Version Kilo $\Rightarrow$ Pike
	\item libvirt/QEMU-Instanzen
	\item Nova Docker $\Rightarrow$ Zun
	\item Container-angepasste DevStack-Konfiguration
	\item Vollständige Netzwerkkonfiguration
	
\end{itemize}

\noindent Größte Hürde ist die Limitierung auf einen Prozess innerhalb eines Standard-Docker-Containers. Neuere DevStack-Versionen setzen auf \. Daher muss dies über die Umgebungsvariable \emph{ENV container docker} bekannt gemacht werden. Anschließend lässt sich \emph{systemd} über zwei weitere Workarounds starten\footnote{\url{https://github.com/moby/moby/issues/27202}}\footnote{\url{https://github.com/moby/moby/issues/9212}}.

\emph{Docker Build} bereitet das Image mit allen externen DevStack-Abhängigkeiten vor. Notwendige Dienste wie \emph{RabbitMQ} und \emph{MySQL} werden bereits im Voraus installiert. Das Container-Image führt beim Start nur noch die allerletzten Schritte des Setups aus. Ganz vorweg nehmen lässt sich das Setup nicht, weil während des Builds keine erweiterten Rechte vorliegen.

Nach erfolgreichem Start reicht das Kommando \emph{make run}, um per Zun einen \emph{Cirros}-Basis-Container\footnote{\url{https://docs.docker.com/samples/library/cirros/}} zu starten. Der Stand der gesamten OpenStack-Installation lässt sich per \emph{docker commit} oder experimentell per \emph{Docker-Snapshots}\footnote{\url{https://criu.org/Docker}} sichern.

Anpassbar sind im Skript OpenStack-Services und -Versionen, da DevStack direkt aus den Quellen installiert wird. So ändern sich allerdings selbst die Abhängigkeiten der als stabil gekennzeichneten Versionen. Das Prinzip Infrastruktur als Code geht hier nicht immer auf -- DevStack ist nicht zuverlässig reproduzierbar. \autoref{fig:devstack} vergleicht die Installationsvarianten.

Als \emph{Proof-of-Concept} ist die Integration von Docker, DevStack und Zun bisher einmalig. Der Code ist daher auf GitHub\footnote{\url{https://github.com/janmattfeld/DockStack}} veröffentlicht und zeigt einige \emph{Best Practices} und \emph{Lessons Learned} in Bezug auf die genannten Projekte.

Letztendlich greifen wir auf eine lokale \emph{Mirantis}-OpenStack-Installation aus dem \emph{SSICLOPS}-Projekt zurück. Die Infrastruktur ist virtuell und wird durch \emph{Fuel}\footnote{\url{https://www.mirantis.com/software/openstack/}} zuverlässiger wieder aufgebaut. Die Zun-Container-Dienste sind nicht enthalten; dafür aber alle anderen Kernfunktionen und APIs.
