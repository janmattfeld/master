\section{Private-Cloud: OpenStack-Testbed}
\label{sec:openstack-testbed}

Als Beispiel für eine Private-Cloud -- als Teil unseres Multi-Cloud-Setups -- soll OpenStack dienen. Es ist das populärste Open-Source-Projekt um eigene Infrastruktur als Service aufzubauen. Gesponsert wird es von Großunternehmen wie \emph{HPE}, \emph{IBM}, \emph{Canonical}, \emph{Red Hat} und anderen.

OpenStack setzt sich aus verschiedenen Teilprojekten zusammen, die jeweils einen Dienst entwickeln und bereitstellen. Ein Minimal-Setup besteht aus \emph{Nova} (Computing), \emph{Key\-stone} (Authentifizierung), \emph{Neutron} (Netzwerk) und \emph{Glance} (Images). Verbreitet sind außerdem \emph{Cinder} (Blockspeicher) und \emph{Horizon} (Dash\-board). Diese sollen auch in unserem Beispiel genutzt werden. Denkbar ist darüber hinaus die Integration eines Container-Dienstes. Der Zugriff auf die Infrastruktur erfolgt entweder über das Dashboard, Kommandozeilentools oder eine REST-API.

Grundsätzlich wäre auch der Aufbau einer OpenStack-Föderation wie in \emph{SSICLOPS} denkbar \cite{ssiclops:2015:d6.1-project-presentation}. Föderierte Cloud-Architekturen teilen sich zentrale Komponenten, in OpenStack mindestens den Authentifizierungsservice \emph{Keystone}. Je nach Föderationsvariante (\emph{Cells}, \emph{Regions}, \emph{Availability Zones} oder \emph{Host Aggregates}) sind auch Dienste wie Dashboard oder Speicher nur einmal vorhanden. Diese Architektur reduziert Fixkosten, erfordert allerdings spezielle Anpassungen innerhalb der Cloud. Auch gehen einige Vorteile wie Ausfallsicherheit und Unabhängigkeit der zentralen Dienste wieder verloren. Eine Kombination mit weiteren Cloud-Providern im Rahmen unseres Multi-Cloud-Setups ist denkbar, bleibt aufgrund der aufwendigen Einrichtung aber außen vor. Auch wäre der zusätzliche Erkenntnisgewinn gering.

Selbst ein minimales OpenStack-Testsetup ist durch die diversen Dienste komplex. Denkbar wäre also auch die Nutzung von externen OpenStack-Angeboten. In diesem Projekt gibt es hierfür grundsätzlich drei mögliche Bereitstellungsmodelle: 

\begin{enumerate}
	\item Public Cloud
	\\\emph{Betrieb auf geteilter Cloud-Infrastruktur}
	
	\item Hosted Private Cloud
	\\\emph{Betrieb auf exklusiver Cloud-Infrastruktur}
	
	\item Lokale Testinstallation
	\\\emph{Betrieb auf eigener physischer oder virtueller Infrastruktur}
\end{enumerate}

\noindent Eine Liste öffentlicher OpenStack-Angebote findet sich auf der Projekthomepage\footnote{\url{https://www.openstack.org/marketplace/hosted-private-clouds/}}. Dort werden auch weitere Informationen wie Funktionsumfang und Zertifizierungen aufgeführt. 

Interessant ist zum Beispiel das Angebot der Deutschen Telekom \emph{Open Telekom Cloud\footnote{\url{https://cloud.telekom.de/en/infrastructure/open-telekom-cloud/}}}: eine Public Cloud auf OpenStack-Basis -- in Deutschland -- mit vollem Funktionsumfang und API-Zugriff. International bietet \emph{Rackspace} eine Hosted Private Cloud\footnote{\url{https://www.rackspace.com/openstack/}}. Beide eignen sich jedoch kaum, um kleine Experimente zu starten, sondern richten sich vor allem preislich an größere Organisationen und Unternehmen.

Kostenlos ist das Public-Cloud-Angebot \emph{TryStack}\footnote{\url{http://trystack.org/}}. Sponsoren wie \emph{Cisco}, \emph{NetApp}, \emph{Dell} und \emph{Red Hat} finanzieren das Projekt. Die Registrierung erfolgt über die Aufnahme in eine Facebook-Gruppe, anschließend soll hierüber auch der Zugang zur kostenlosen OpenStack-\emph{Liberty}-Instanz erfolgen. Während der gesamten Laufzeit dieser Arbeit war allerdings weder ein Login noch Kontakt zu den Organisatoren möglich.

Lokale OpenStack-Installationen sind aufwendig: Für jeden Dienst muss ein eigener physikalischer Rechner bereitstehen. Dementsprechend verweist die offizielle Dokumentation direkt auf die Vielzahl von OpenStack-Distributionen\footnote{\url{https://www.openstack.org/marketplace/distros/}}. Diese bieten fast immer einen vereinfachten Setup-Prozess und oft die Option statt physikalischen Rechnern virtuelle Maschinen oder Container zu nutzen. Wie auch bei den Hosted-Angeboten sind hier nicht alle Dienste verfügbar. In allen Paketen fehlt \emph{Zun}, der aktuelle Container-Service.

Speziell für lokale Test- und Entwicklungsumgebungen existiert \emph{DevStack}\footnote{\url{https://docs.openstack.org/devstack/latest/}}. Das offizielle OpenStack-Projekt installiert automatisiert die wichtigsten OpenStack-Dienste auf einer einzigen Maschine. Ausdrücklich unterstützt werden dabei auch VMs und \emph{LXC}-Container. Es soll daher als Erstes erprobt werden.