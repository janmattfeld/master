\section{Public-Cloud: AWS-Testbed}
\label{sec:aws-testbed}

\todo{AWS-Intro} Die Amazon-Web-Services-Integration ist eine besondere Herausforderung: Der komplette Auf- und Abbau der Testinfrastruktur benötigt mehrere Minuten. Zusätzlich kostet jede Ressourcennutzung Geld, unabhängig vom Zweck. Zwar stellt Amazon einmalig 100 Dollar Testbudget zur Verfügung; bei mehreren Hyrise-Instanzen sind diese jedoch innerhalb weniger Tage aufgebraucht. Durch die komplexe Struktur der Amazon-Dienste entsteht -- gerade während der Entwicklung des Cloud-Adapters -- ein hohes Kostenrisiko. Der Amazon-Container-Dienst ECS setzt immer eine gestartete Instanz der Elastic Compute Cloud (EC2) voraus. Diese verursacht auch Kosten, wenn kein Container ausgeführt wird. Damit ist das AWS-Verhalten nicht intuitiv und steht in Gegensatz zur \emph{On-Demand}--Philosophie.

Um Aufwand und Risiko zu senken, eignen sich nicht-funktionale Platzhalter-Bibliotheken, sogenannte \emph{Mocks}. Sie implementieren die gleichen Schnittstellen wie herstellereigene Originale und beantworten Anfragen entsprechend -- jedoch ohne tatsächlich eine Netzwerkverbindung und entfernte Ressourcen zu nutzen. Für AWS existieren zwei interessante Projekte; \emph{Moto}\footnote{\url{http://docs.getmoto.org/en/latest/}} und \emph{LocalStack}\footnote{\url{https://localstack.cloud/}}. 

Moto ist eine Python-Bibliothek, die AWS-Aufrufe per Decorator oder Context-Manager direkt im Quellcode umleitet. Der angedachte Anwendungsfall ist die Kombination mit \emph{boto}, dem offiziellen AWS-Python-SDK\footnote{\url{https://aws.amazon.com/de/sdk-for-python/}}.

LocalStack ist  bibliotheks- und programmiersprachen-unabhängig. Es erstellt HTTP-Endpunkte für alle AWS-Dienste. Änderungen im Quellcode der eigentlichen Anwendung sind dafür nicht nötig. Daher ist auch eine Kombination mit Libcloud denkbar, um den Aufwand der AWS-Integration erheblich zu senken. Insgesamt erfordert die Arbeit mit AWS viel Aufmerksamkeit -- die wir lieber in die Konzeption des Brokers investieren. Im lokalen Test- und Entwicklungs-Set-up nutzen wir deshalb Docker als Amazon-ECS-Ersatz.

Für OpenStack sind keine derartig ausgereiften Mock-Projekte bekannt. Alle offiziellen Bibliotheken dienen eher zum Test der OpenStack-Komponenten selbst. Eine tatsächliche Ausführung von Gastanwendungen ist nicht vorgesehen. Um die Funktion mit OpenStack zu validieren, erstellen wir später unser eigenes Testbed im \autoref{sec:openstack-testbed}. Längerfristig wollen wir von Libcloud profitieren, da der Implementierungsaufwand nur einmalig anfällt. Anschließend existiert die Abstraktionsebene unabhängig von potenziellen Schnittstellenänderungen eines Cloud-Providers.