\null\vfil
\begin{otherlanguage}{ngerman}
\begin{center}\textsf{\textbf{\abstractname}}\end{center}

Cloud-Ressourcen spielen eine entscheidende Rolle in den aktuellen und zukünftigen IT-Strategien von Unternehmen aller Größen. Gleichzeitig entstehen zentrale Herausforderungen in Hinblick auf Vertraulichkeit und Zuverlässigkeit, sowie Portabilität der eigenen Daten und Anwendungen. Mit dem Inkrafttreten der Datenschutz-Grundverordnung und immer neuen Datenlecks ist das Thema hochaktuell.

Durch die intelligente Kombination von Private- und Public-Cloud-Infrastruktur treten wir diesen Herausforderungen entgegen. Wir geben einen Überblick über aktuelle kommerzielle und akademische Multi-Cloud-Projekte. Dabei identifizieren wir fehlende SLA- und Cloud-Schnittstellen-Standards als größte Risiken.

Als Lösungsvorschlag entwickeln wir einen Multi-Cloud-Anwendungs-Broker auf den Ebenen IaaS und CaaS. Dabei dient das TOSCA-Simple-Schema zur Anwendungs- und SLA-Spezi\-fi\-ka\-tion. Wir diskutieren den Aufwand der Cloud-Teststellungen und den Einsatz der Multi-Cloud-Bibliothek Apache Libcloud. Die Leistungsfähigkeit der Lösung demonstrieren wir anschließend in einem Testaufbau mit OpenStack, AWS, Docker und Hyrise-R.

Wir versetzen Cloud-Kunden in die Lage, ihre Anforderungen anbieter- und technikunabhängig zu formulieren und durchzusetzen. Unsere Lösung automatisiert die Einhaltung von Datenschutz-, Qualitäts- und Kostenzielen. Damit ist das Projekt einzigartig und ergänzt die bisherigen Beiträge im SSICLOPS-Kontext.

\end{otherlanguage}
\vfil\null



