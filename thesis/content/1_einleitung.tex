\chapter{Einleitung}

Cloud-Angebote sind allgegenwärtig und werden mittlerweile von einer Mehrzahl deutscher Unternehmen genutzt. Laut \emph{Statista} setzten bis 2016 mindestens fünfundsechzig Prozent aller Betriebe entsprechende Lösungen ein \cite{bitkom:2017:cloud-nutzung-unternehmen}. Besonders gefragt sind Infrastrukturdienste, wie Rechenleistung und Speicher. Gleich darauf folgen Softwareangebote und E-Mail-Hosting. Etwas abgeschlagen bleiben Plattformdienste wie Datenbanken und Ausführungsumgebungen \cite{destatis:2016:cloud-nutzung-unternehmen-einsatzzweck}. Der Trend zur Cloud wird sich vermutlich fortsetzen: So vervierfacht sich das prognostizierte Marktvolumen mit Cloud-Services bis 2020 auf über sechzehn Milliarden Euro allein im deutschen B2B-Markt \cite{isg:2017:cloud-ausgaben-2020}.

Cloud-Angebote sind für Unternehmen aller Größen attraktiv: Die Anschaffung eigener Infrastruktur entfällt, genauso wie deren Wartung durch eigenes Personal. Stattdessen lassen sich Ressourcen und Anwendungen einfach per Self-Service buchen und sind anschließend über das Internet von überall erreichbar. Der Umfang gebuchter Leistungen lässt sich meist frei skalieren. Da die Angebote oft in kleinem Takt und verbrauchsgenau abgerechnet werden, ergeben sich so theoretisch Vorteile bei Flexibilität und Kosteneffizienz.

Demgegenüber stehen Vorbehalte bezüglich Datenschutz, denn die gemietete Infrastruktur teilen sich mehrere Kunden. Durch Sicherheitslücken wie \emph{Spectre} und \emph{Meltdown} werden eigentlich geschützte Speicherbereiche angreifbar \cite{Kocher2018spectre, Lipp2018meltdown}. Nach aktuellem Stand existierten diese Lücken mehr als ein halbes Jahr in fast jedem x86-System, und damit in fast jeder Cloud \cite{techcrunch:2018:spectre-meltdown-tier-2-cloud-vendors}. Eine sichere Mandantentrennung war also nicht mehr gewährleistet \cite{aws:2018:security-bulletin}. Aber auch durch Unachtsamkeit werden Cloud-Datenspeicher immer wieder der Öffentlichkeit zugänglich \cite{upguard:2017:breach-alteryx, upguard:2017:breach-centcom, kromtech:2018:breach-fedex}.

Zugleich fordert aktuelle Gesetzgebung wie die Datenschutz-Grundverordnung unter anderem die Verarbeitung von personenbezogenen Daten europäischer Kunden ausschließlich innerhalb der EU \cite{eu:2016:bdsvg}. Gut neunzig Prozent aller Unternehmen achteten folglich bei der Auswahl eines Cloud-Providers auf Rechts- und Server-Standorte in Deutschland \cite{gartner:2017:cloud-market}. Auch vorhandene Softwarelizenzen können den Umzug verhindern. Zum Beispiel Oracle- und Microsoft-OEM-Lizenzen verbieten die Übertragung in eine Cloud-Umgebung \cite{microsoft:2017:licensing, oracle:2018:licensing}. So müssen möglicherweise Teile der Infrastruktur lokal vorgehalten werden.

Für gut fünfzig Prozent der interessierten Unternehmen außerdem unabdingbar: individuelle Dienstgütevereinbarungen, sogenannte \emph{Service Level Agreements (SLAs)} \cite{bitkom:2017:cloud-nutzung-unternehmen-auswahlkriterien}. Dies lässt jedoch einige der weltweit größten Cloud-Anbieter außen vor -- Amazon und Microsoft teilen sich seit 2016 über fünfzig Prozent des weltweiten Umsatzes mit Infrastrukturdiensten \cite{gartner:2017:cloud-market}. Die Vertragsbedingungen beider Anbieter sind jedoch nicht verhandelbar.

Laut Gartner wollen daher 70 Prozent aller Unternehmen bis 2019 eine Multi-Cloud-Strategie umsetzen \cite{gartner:2017:cloud-market-multicloud-trend}. Auch das EU-geförderte Forschungsprojekt \emph{Scalable and Secure Infrastructures for Cloud Operations (SSICLOPS)} identifiziert Multi-Cloud-Umgebungen als zukünftigen Treiber des ITK-Markts \cite{ssiclops:2015:d6.1-project-presentation}. Zusätzlich zu einer privaten Cloud werden hierbei auch Dienste aus weiteren öffentlichen Angeboten genutzt. Vorteil ist eine höhere Flexibilität, um für jede Anforderung die ideale Cloud wählen zu können. Kriterien sind zum Beispiel die Einhaltung gesetzlicher Anforderungen, spezielle SLAs, höhere Ausfallsicherheit und die Preisgestaltung der Anbieter. Multi-Cloud birgt aber auch einige Herausforderungen:

\begin{itemize}
	\item Portabilität eigener Anwendungen
	\item Cloud-Provider-spezifisches Know-how
	\item Höherer IT-Verwaltungsaufwand der Ressourcen
	\item Managementaufwand, wie der Vergleich von Compliance-Richtlinien
	\item Keine oder geringere preisliche Skaleneffekte
	\item Überwachung der heterogenen Cloud-Landschaft
	\item Durchsetzung der eigenen Sicherheits- und Datenschutzrichtlinien
\end{itemize}

Um diese Herausforderungen automatisiert abzumildern, eignet sich eine \emph{Cloud-Management-Plattform (CMP)} mit integriertem Anwendungs-Broker. Deren Marktsituation ist allerdings unübersichtlich und in großer Bewegung. Bereits im letzten Jahr wurden einige vielversprechende Lösungen von Cloud-Infrastruktur-Providern aufgekauft \cite{gartner:2017:cloud-market-multicloud-trend}. Die CMPs sind nun selbst Software-as-a-Service-Angebote und proprietär. Die eigentlichen Vorteile Flexibilität und Unabhängigkeit werden so ad absurdum geführt.

Unabhängige Lösungen sind mehrheitlich ausgelaufene akademische Forschungsprojekte. Auf den aktuellen Cloud-Markt und technische Neuerungen wie Container sind diese nicht mehr anwendbar. Aufgrund der vorherigen Forschungsergebnisse ergibt sich jedoch folgende Hypothese:

\begin{verse}
	{\emph{Eine unabhängige, auf offenen Standards basierende CMP kann die \newline Vorteile der Cloud mit aktuellem Datenschutz vereinen.}}
\end{verse}

Im Rahmen dieses Projektes entwickeln wir einen Multi-Cloud-Broker. Alle Ergebnisse basieren auf Open-Source-Technologien und sollen, wenn möglich, an die Ursprungsprojekte zurückfließen. Der Broker soll Organisationen und Unternehmen als Cloud-Nutzer emanzipieren und besonders den kommerziellen Wert der Lösung herausstellen.

Da SLAs und weitere Rahmenbedingungen oft nicht verhandelbar sind, muss die Optimierung der Cloud-Nutzung in Eigenregie erfolgen.  Die vorgestellte CMP erlaubt die Nutzung bestimmter Anbieter, obwohl SLAs  nicht eingehalten werden: Zum Beispiel lässt sich eine höhere Ausfallsicherheit durch Kombination mehrerer Anbieter erreichen. Nebenbei entsteht so ein Notfallplan zum Weiterbetrieb bei Ausfall oder Kündigung eines Cloud-Providers.

Das folgende Kapitel klassifiziert Cloud-Angebote anhand bestimmter Eigenschaften und Service-Ebenen. Außerdem geben wir eine Einführung in datenschutzrechtliche Fragen und Risiken in Zusammenhang mit Cloud-Bereit\-stel\-lungs\-model\-len.

Anschließend entwickeln wir passende Policys und entsprechende Schemata. Diese sollen von dem eigens entwickelten Multi-Cloud-Broker verwendet werden. Er verteilt auch bestehende, nicht Cloud-native, Anwendungen über verschiedene Cloud-Provider auf IaaS- und CaaS-Ebenen und optimiert die Cloud-Landschaft anschließend fortlaufend. Dabei beachtet er SLAs und Datenschutzanforderungen. Für diesen Vorschlag bewerten wir akademische und kommerzielle Projekte mit ähnlicher Zielsetzung.

Der Implementierungsteil vergleicht verschiedene Multi-Cloud-Bibliotheken. Auf Basis von Apache \emph{Libcloud} entwickeln wir eine CMP, die das Multi-Cloud-Brokering übernimmt. Auch Implementierungshürden und Besonderheiten der verschiedenen Clouds werden beschrieben. Als Beispielanwendung dient dabei die verteilte Forschungsdatenbank \emph{Hyrise-R}. Auch die Entstehung eines \emph{OpenStack}-Testbeds als Beispiel einer Private-Cloud wird besprochen. Es folgt eine abschließende Bewertung des Konzepts.
