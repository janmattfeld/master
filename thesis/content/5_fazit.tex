%In der Schlussbetrachtung gibt es einen Rückblick, in dem Motivation und Thesen aus der Einleitung wieder aufgegriffen und abgerundet werden. Antworten auf in der Problemstellung aufgeworfene Fragen werden kurz und prägnant zusammengefasst. Ebenso sollte ein Ausblick auf offen gebliebene Fragen sowie auf interessante Fragestellungen, die sich aus der Arbeit ergeben, gegeben werden. Ein persönlich begründetes Fazit aus eigener Perspektive ist an dieser Stelle ebenfalls sinnvoll.
%
%Die Motivation darf nicht auf den Abstract aufbauen, sondern als Startpunkt der Ausarbeitung dienen.
%Die zentralen Fragen aufzulisten, die im Rahmen der Arbeit beantwortet werden.
%Es muss für den nicht an der Durchführung der Arbeit beteiligten Leser verständlich und nachvollziehbar sein – lieber etwas mehr Kontext und Hintergrundwissen vermitteln.
%
%Das Problem und die Forschungsfragestellung als relevant und aktuell darstellen – ggf. mit Verweis auf aktuelle Veröffentlichungen, Vorträge, Geschehnisse oder Presseberichte.
%Ergebnisse der Arbeit dürfen auch hier erwähnt werden.
%Wenn nötig, kann die Terminologie hier schon eingeführt werden.
%Die Problemdomäne kurz anreißen und ggf. auf weiterführende Literatur verwiesen.
