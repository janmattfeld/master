\documentclass{scrreprt} % <= Druckversion: "scrbook", Bildschirmversion: "scrreprt"
\newcommand\bcor{12mm} % <= Bindungskorrektur für Druckversion
\usepackage{osm-thesis}

% ABOUT
\newcommand{\hpitype}{Masterarbeit}
\newcommand{\hpiauthor}{Jan-Henrich Mattfeld}
\newcommand{\hpititle}{Eine einheitliche Middleware zur Policy-Durchsetzung in Multi-Cloud-Infrastrukturen}
\newcommand{\hpititleother}{A Unified Middleware for Policy Enforcement in Multi-Cloud Infrastructures} % <= das Studienreferat verlangt einen deutschen UND englischen Titel
\newcommand{\hpisupervisor}{Max Plauth, Prof.\,Dr.\,Andreas Polze}
\newcommand{\hpichair}{Fachgebiet für Betriebssysteme und Middleware}
\newcommand{\hpiexternalsupervisor}{}
\newcommand{\hpiexternal}{}
\newcommand{\hpidate}{\today}

% DOCUMENT
%\KOMAoption{draft}{true} % <= z.B. zum "Debuggen" der Overfull-Boxes
\bibliography{bibliography}

\begin{document}
	\selectlanguage{ngerman}

	% Einband
	\pagenumbering{alph}
	\ifisbook\include{content/coverpage}\fi
	\ifisbook\cleardoubleemptypage\fi

	% (Haupt-)Titelseite, Abstract, ggf. Danksagung & Inhaltsverzeichnis
	\pagenumbering{roman}
	\include{content/titlepage}
%	\ifisbook\cleardoubleemptypage\fi\null\vfil
\begin{otherlanguage}{ngerman}
\begin{center}\textsf{\textbf{\abstractname}}\end{center}

Cloud-Ressourcen spielen eine entscheidende Rolle in den aktuellen und zukünftigen IT-Strategien von Unternehmen aller Größen. Gleichzeitig entstehen zentrale Herausforderungen in Hinblick auf Vertraulichkeit und Zuverlässigkeit, sowie Portabilität der eigenen Daten und Anwendungen. Mit dem Inkrafttreten der Datenschutz-Grundverordnung und immer neuen Datenlecks ist das Thema hochaktuell.

Durch die intelligente Kombination von Private- und Public-Cloud-Infrastruktur treten wir diesen Herausforderungen entgegen. Wir geben einen Überblick über aktuelle kommerzielle und akademische Multi-Cloud-Projekte. Dabei identifizieren wir fehlende SLA- und Cloud-Schnittstellen-Standards als größte Risiken.

Als Lösungsvorschlag entwickeln wir einen Multi-Cloud-Anwendungs-Broker auf den Ebenen IaaS und CaaS. Dabei dient das TOSCA-Simple-Schema zur Anwendungs- und SLA-Spezi\-fi\-ka\-tion. Wir diskutieren den Aufwand der Cloud-Teststellungen und den Einsatz der Multi-Cloud-Bibliothek Apache Libcloud. Die Leistungsfähigkeit der Lösung demonstrieren wir anschließend in einem Testaufbau mit OpenStack, AWS, Docker und Hyrise-R.

Wir versetzen Cloud-Kunden in die Lage, ihre Anforderungen anbieter- und technikunabhängig zu formulieren und durchzusetzen. Unsere Lösung automatisiert die Einhaltung von Datenschutz-, Qualitäts- und Kostenzielen. Damit ist das Projekt einzigartig und ergänzt die bisherigen Beiträge im SSICLOPS-Kontext.

\end{otherlanguage}
\vfil\null




	%\ifisbook\cleardoubleemptypage\fi\include{content/dedication}
%	\tableofcontents
	\cleardoublepage

	% Textteil
	\pagenumbering{arabic}
	\chapter{Einleitung}

% Im Zentrum der Einleitung stehen die Vorstellung und Motivation des Themas der Arbeit und die genaue Auflistung der Fragestellungen (Wieso ist das Thema relevant?). Ebenso sollten schon einzelne Aspekte des Problems herausgearbeitet werden. Dabei ist es hilfreich, die zentralen Fragen aufzulisten, die im Rahmen der Arbeit beantwortet werden sollen. Außerdem sollte ein knapper Überblick gegeben werden, in welchen Schritten die Problembehandlung erfolgt: Hinführung zum Thema, Herleitung und Ausformulierung der Fragestellung, Abgrenzung des Themas (Angabe von Aspekten, die zum Thema gehören, aber ausgeklammert werden) und Aufbau der Arbeit (Begründung der Gliederung).

Cloud-Angebote sind allgegenwärtig und werden mittlerweile von der Mehrzahl deutscher Unternehmen genutzt. Laut Statista setzten bis 2016 mindestens fünfundsechzig Prozent aller Betriebe entsprechende Lösungen ein. 
%https://de.statista.com/statistik/daten/studie/177484/umfrage/einsatz-von-cloud-computing-in-deutschen-unternehmen-2011/
Besonders gefragt sind Infrastrukturdienste, wie Rechenleistung und Speicher. Gleich darauf folgen Softwareangebote und E-Mail-Hosting. Etwas abgeschlagen bleiben Plattformdienste wie Datenbanken und Ausführungsumgebungen. %https://de.statista.com/statistik/daten/studie/381830/umfrage/einsatzzwecke-von-cloud-computing-in-unternehmen-in-deutschland/
Der Trend zur Cloud wird sich vermutlich fortsetzen: So vervierfacht sich das prognostizierte Marktvolumen mit Cloud-Services bis 2020 auf über sechzehn Milliarden Euro allein im deutschen B2B-Markt. %
%https://de.statista.com/statistik/daten/studie/165458/umfrage/prognostiziertes-marktvolumen-fuer-cloud-computing-in-deutschland/

Cloud-Angebote sind für Unternehmen aller Größen attraktiv: Die Anschaffung eigener Infrastruktur entfällt, genauso wie deren Wartung durch eigenes Personal. Stattdessen lassen sich Ressourcen und Anwendungen einfach per Self-Service buchen und sind anschließend über das Internet von überall erreichbar. Der Umfang gebuchter Leistungen lässt sich meist frei skalieren. Da die Angebote oft in kleinem Takt und verbrauchsgenau abgerechnet werden, ergeben sich so theoretisch Vorteile bei Flexibilität und Kosteneffizienz.

Demgegenüber stehen Vorbehalte bezüglich Datenschutz, denn die gemietete Infrastruktur teilen sich mehrere Kunden. Technische Fehler wie der 
%https://aws.amazon.com/security/security-bulletins/AWS-2018-013/
% Mittelgroße Provider wurden gar nicht erst informiert: https://techcrunch.com/2018/01/06/how-tier-2-cloud-vendors-banded-together-to-cope-with-spectre
Intel-Prozessor-Bug lassen eigentlich geschützte Speicherbereiche angreifbar werden. Nach aktuellem Stand existierte diese Lücke mehr als ein halbes Jahr in fast jedem x86-System, und damit in fast jeder Cloud. Eine sichere Mandantentrennung war also nicht mehr gewährleistet. Aber auch durch Unachtsamkeit werden Cloud-Datenspeicher immer wieder der Öffentlichkeit zugänglich.
%Amazon S3 Datenlecks

Zugleich fordert aktuelle Gesetzgebung wie die Datenschutz-Grundverordnung unter anderem die Verarbeitung von personenbezogenen Daten europäischer Kunden ausschließlich innerhalb der EU. %http://data.consilium.europa.eu/doc/document/ST-12399-2016-INIT/en/pdf %Anbieter sind in der Pflicht. Haftung auch Gegenüber Endkunden, mit denen eigentlich keine direkte Geschäftsbeziehung besteht. Löschpflicht. Wo sind die Daten physikalisch? Portabilität Cloud->Kund und Cloud->Cloud
%Weitere Anforderungen Diese Regelungen sind keineswegs auf Europa begrenzt. Gleiches in der USA (Amazon Gov CLoud)
Gut neunzig Prozent aller Unternehmen achteten dementsprechend bei der Auswahl eines Cloud-Providers auf Rechts- und Server-Standorte in Deutschland. 

Auch vorhandene Softwarelizenzen können den Umzug verhindern. Zum Beispiel Oracle oder Microsoft-OEM-Lizenzen erlauben die Übertragung in eine Cloud-Umgebung nicht. So müssen möglicherweise Teile der Infrastruktur lokal vorgehalten werden.



Für gut fünfzig Prozent der interessierten Unternehmen außerdem unabdingbar: individuelle SLAs.
%https://de.statista.com/statistik/daten/studie/545924/umfrage/kriterien-bei-der-auswahl-eines-cloud-providers-in-deutschen-unternehmen/
Dies lässt jedoch einige der weltweit größten Cloud-Anbieter außen vor -- Amazon und Microsoft teilen sich seit 2016 über fünfzig Prozent des weltweiten Umsatzes mit Infrastrukturdiensten. Die Vertragsbedingungen beider Anbieter sind jedoch meist nicht verhandelbar. %https://de.statista.com/statistik/daten/studie/754647/umfrage/marktanteile-am-umsatz-mit-infrastructure-as-a-service-weltweit/

Laut Gartner wollen daher 70 Prozent aller Unternehmen bis 2019 eine Multi-Cloud-Strategie umsetzen. Zusätzlich zu einer privaten Cloud werden hierbei auch Dienste aus weiteren öffentlichen Angeboten genutzt. Vorteil ist eine höhere Flexibilität, um für jede Anforderung die passendste Cloud wählen zu können. Kriterien sind zum Beispiel die Einhaltung gesetzlicher Anforderungen, spezielle SLAs, höhere Ausfallsicherheit und die Preisgestaltung der Anbieter. Multi-Cloud birgt aber auch einige Herausforderungen:

% XKCD https://xkcd.com/1938/


% Capgemini WHitepaper "from boring to sexy" Multi Cloud ist Trend."
%
%https://www.computerwoche.de/a/migration-in-die-cloud-so-vermeiden-sie-fehler,3327477
%
%https://www.computerwoche.de/a/der-trend-geht-zur-multi-cloud,3332352
%
%https://www.computerwoche.de/a/mit-der-multi-cloud-zur-digitalen-infrastruktur,3331691
%
%https://www.computerwoche.de/a/was-ist-multi-cloud-der-naechste-schritt-im-cloud-computing,3331658
%
%https://www.gartner.com/doc/reprints?id=1-4KKGOTA&ct=171115&st=sb%3fsrc=so_5703fb3d92c20&cid=70134000001M5td

\begin{itemize}
	\item Portabilität eigener Anwendungen
	\item Cloud-Provider-spezifisches Know-How
	\item Höherer IT-Verwaltungsaufwand der Ressourcen
	\item Keine oder geringere (preisliche) Skaleneffekte
	\item Managementaufwand, wie der Vergleich der Compliance-Richtlinien
	\item Überwachung der heterogenen Cloud-Landschaft
	\item Durchsetzung der eigenen Sicherheits- und Datenschutzrichtlinien
\end{itemize}

\noindent
Um diese Herausforderungen automatisiert abzumildern, eignet sich eine Cloud Management Plattform (CMP). Deren Marktsituation ist allerdings unübersichtlich und in großer Bewegung. Bereits im letzten Jahr wurden einige vielversprechende Lösungen von Cloud-Infrastruktur-Providern aufgekauft. Die CMPs sind nun selbst Software-as-a-Service-Angebote und proprietär. Die eigentlichen Vorteile Flexibilität und Unabhängigkeit werden so ad absurdum geführt.

Tatsächlich unabhängige Lösungen sind mehrheitlich ausgelaufene akademische Forschungsprojekte. Auf den aktuellen Cloud-Markt und technische Neuerungen wie Container sind diese nicht mehr anwendbar. Aufgrund der vorherigen Forschungsergebnisse ergibt sich jedoch folgende Hypothese:

%CMPs ändern sich ständig. selbst während dieser arbeit sind anbieter ausgeschieden, aufgekauft oder grundlegend verändert. cost als neues feature. kein echtes brokering, sondern self-service und monitoring.

\begin{verse}
{Eine unabhängige, auf offenen Standards basierende CMP kann die Vorteile der Cloud mit aktuellem Datenschutz zu vereinen.}
\end{verse}

\noindent Die Verwaltung der verteilten Applikationen sollte weiterhin automatisch erfolgen. Da SLAs und weitere Rahmenbedingungen oft nicht verhandelbar sind, muss die Optimierung der Cloud-Nutzung in Eigenregie erfolgen. Dabei sollen Organisationen und Unternehmen als Cloud-Nutzer emanzipiert werden. Die vorgestellte CMP erlaubt die Nutzung bestimmter Anbieter, obwohl SLAs  nicht eingehalten werden: zum Beispiel lässt sich eine höhere Ausfallsicherheit durch Kombination mehrerer Anbieter erreichen. Nebenbei entsteht so ein Notfallplan zum Weiterbetrieb bei Ausfall oder Kündigung eines Cloud-Providers.

Das folgende Kapitel klassifiziert Cloud-Angebote anhand bestimmter Eigenschaften und Service-Ebenen. Außerdem geben wir eine Einführung in datenschutzrechtliche Fragen und Risiken in Zusammenhang mit Cloud-Bereit\-stel\-lungs\-model\-len.

Anschließend entwickeln wir passende Policies und entsprechende Schemata. Diese sollen von einem eigens entwickelten Multi-Cloud-Broker verwendet werden. Er verteilt auch nicht cloud-native Legacy-Anwendungen über verschiedene Cloud-Provider auf IaaS- und CaaS-Ebenen und optimiert die Cloud-Landschaft anschließend fortlaufend. Dabei beachtet er SLAs und Datenschutzanforderungen. Für diesen Vorschlag bewerten wir akademische und kommerzielle Projekte mit ähnlicher Zielsetzung.

Der Implementierungsteil vergleicht verschiedene Multi-Cloud-Bibliotheken. Auf Basis von Apache libcloud entwickeln wir eine CMP, die das Multi-Cloud-Brokering übernimmt. Auch Implemtierungshürden und Besonderheiten der verschiedenen Clouds werden beschrieben. Als Beispielanwendung dient dabei die verteilte Forschungsdatenbank Hyrise-R. Auch die Entstehung des OpenStack-Testbeds als Beispiel einer Private-Cloud wird besprochen.

Es folgt eine abschließende Bewertung des Konzepts -- kann eine unabhängige CMP die Hürden der Cloud-Nutzung durch einen Multi-Cloud-Ansatz senken?

%- Klassifizierung von Cloud-Angebote und Risiken
%- Datenschutzrechtliche Bewertung von Daten 
%- Portabilität von Anwendungen für einfache Cloud-übergreifende Migrationen
%- Entwicklung von entsprechenden Policies und einem Schema
%- Vorschlag eines eigens entwickelten Multi-Cloud-Brokers, der auch nicht cloud-native Anwendungen über verschiedene Cloud-Provider auf IaaS, CaaS und PaaS-Ebenen in Hybriden Umgebungen verteilt und managt. Dabei beachtet er SLAs und Datenschutzanforderungen.
%- Related work: Gegenüberstellung Verschiedener kommerzieller und akademischer Projekte mit ähnlicher Zielsetzung. Fehlende Eigenschaften
%Wie nützlich ist Apache libcloud (oder generell eine Multi-Cloud Library? Vergleich!)
%- Einbindung bisher nicht cloud-nativer Anwendungen am Beispiel von hyrise-R
%- Technische Betrachtung des Prototypen mit Hyrise-R und OpenStack als Grundlage im Rahmen von SSCICLOPS
%- Future Work und Bewertung des Prototypen
%
%Nur Teile einer CMP umgesetzt. Identity spielt hier z.\,B. keine Rolle. 
	% indem sie beispielsweise aus laufenden oder abgeschlossenen wissenschaftlichen Arbeiten bekannte Lösungen vertieft, verbreitert, bestätigt oder widerlegt oder aus einem neuen Blickwinkel betrachtet bzw. an einen neuen Anwendungsfall anpasst.

%Im Grundlagen-Teil sollen zentrale Begriffe definiert und eingeordnet werden. Es geht dabei nicht darum, Definitionen aus Lexika zu suchen; stattdessen sollen problemorientierte Definitionen gegeben werden. Häufig können einzelne Begriffe unterschiedlich weit oder eng definiert werden, sodass auch eine Diskussion unterschiedlicher Definitionsansätze hilfreich sein kann, bevor eine für die weitere Arbeit verbindliche Definition gewählt wird.
%Zudem sollte ein Überblick über die in der Literatur vorhandenen Methoden bzw. Lösungsansätze, der aktuelle Stand der Technik und verwandte Arbeiten gegeben werden. 

%Auch die einzelnen Abschnitte sollten Einleitung, Hauptteil, Schluss enthalten. Die Einleitung und Schluss sollten dabei die Überleitung zu den umschließenden Abschnitten realisieren.




	\chapter{Multi-Cloud-Anwendungs-Broker}
\label{cha:broker}

% Vorteile
% Neue Märkte in anderen Regionen der Welt
% Schnelles Ausrollen neuer Apps
% DevOps geeignet
% Risikoreduzierung
% Reduzierung von (gleichzeitig) Investitionsausgaben und Betriebskosten
% Bestehende Cloud-Deployments mit verwalten-

Um eine Anwendung automatisiert auf verschiedene Clouds zu verteilen ist ein Broker-Mechanismus nötig. Dieser sollte nach verschiedenen, festzulegenden Kriterien vorgehen. Konkret erfüllt ein Broker in der Regel folgende Aufgaben:

\begin{enumerate}
	\item Bereitstellen der Ressourcen für eine Applikation; starten einer virtuellen Maschine oder Reservierung von Speicherplatz
	\item Starten der Anwendung auf den vorher reservierten Ressourcen
	\item Verteilen eingehender Anfragen auf gestartete Anwendungs-Instanzen
	\item Management der Ressourcen
\end{enumerate}

\noindent
Zusätzlich sollte der Multi-Cloud-Broker Policys und SLAs auswerten und umsetzen können. Denkbar ist das automatische Re-provisionieren anhand von 

\begin{enumerate}
	\item Lastspitzen oder Ausfällen von Hardware und Netzwerkressourcen 
	%	(Monitoring)
	\item Geänderten Umfeldparametern wie der Gesetzgebung, Preisen oder AGBs
	\item Nutzeränderungen
	\item Vorherigen Broker-Aktionen
\end{enumerate}

\noindent
In einer Community Cloud könnten Cloud-Provider selbst einen Mechanismus zum Brokering oder zumindest offene APIs bereitstellen. Der Broker wäre also Teil der Cloud. Möglich sind entweder ein zentraler Broker, der direkt auf Cloud-Interna zugreift, oder aber ein Peer-to-Peer-Verbund.\todo{Grafik Architekturübersicht}

Für Multi-Clouds kommen diese Lösungen nicht infrage: Sie bestehen aus mehreren unabhängigen, meist privaten, Cloud-Providern. Aufgrund gegenläufiger Geschäftsinteressen sind diese nicht an einer Föderation mit anderen Anbietern interessiert. Sie werden also weder Interna ihrer Cloud-Plattform anpassen, noch einheitliche APIs bereitstellen.

Stattdessen muss der Broker in einer Multi-Cloud-Umgebung extern bereitgestellt werden. In diesem Fall kann er entweder als eigenständiger Service in Form einer \emph{Cloud Management Platform} angelegt sein, oder von der verteilten Anwendung selbst implementiert werden. Selbst entwickelte CMPs oder integrierte Broker setzen oft auf Multi-Cloud-Bibliotheken wie \emph{Apache libcloud}. \autoref{sec:bibliotheken} bietet hierzu eine Übersicht aktueller Open Source-Projekte. 

Folgende weitere Aspekte sollen bei der Betrachtung der CMPs berücksichtigt werden:

\begin{description}
	
	\item[Zielgruppe] 	\emph{Entwickler und Administratoren}: Im Rahmen von DevOps stellen sie verschiedene Ausführungsumgebungen für Entwicklung, Test und Produktion bereit. Dabei nutzen sie die Self-Service-Funktionen der CMPs.
	
						\emph{Management}: Die Auslastungs- und Kostenübersicht ermöglicht weitere Planung. SLAs und Policys werden überwacht und durchgesetzt.
	
	\item[Anwendungen] Grundsätzlich alle interaktiven Anwendungen und Dienste, sowie Stapelverarbeitungs-Jobs. Diese können verteilt sein, die CMP muss in diesem Fall z.\,B. die Nähe des Datenspeichers zur Rechen-Einheiten beachten.
	
	Ausgenommen spezielle Big Data Analytics und Forschungsszenarien, die unter Umständen besondere Features, Rechte und Architekturen benötigen.
	
	\item[Funktionsumfang] Über die grundlegende Provisionierung hinaus sollte die CMP auch bei weiteren Orchestrationsaufgaben unterstützen: Konfiguration, Monitoring und Skalierung.
%	https://dzone.com/articles/cloud-management-roundup-orchestration-vs-paas-vs-cmp
	
	Die Unterstützung aktueller Container-Technologien und Cloud-Native-Architekturen ist wünschenswert. Keine Rolle spielt \emph{Bare Metal}: Eine installierte Virtualisierungsschicht oder Container-Laufzeitumgebung wird vorausgesetzt. 
	
	Die Auswertung der SLAs und Policys ist auf die verteilten Anwendungen selbst beschränkt. Darüber liegende (CMP-Nutzer) oder tiefergehende Schichten (Service-Nutzer und -Daten) werden extern verwaltet.
		
\end{description}


\noindent
Der folgende Abschnitt gibt eine Übersicht kommerzieller Cloud Management Plattformen sowie bisheriger Forschung zu Inter- und Multi-Cloud-Brokern mit besonderem Blick auf SLAs und Policys. Die vorgestellten Lösungen unterscheiden sich in Architektur, Flexibilität und Funktionsumfang. Die vier Broker-Basiseigenschaften werden nicht von allen Arbeiten in vollem Umfang erfüllt.

Weiterhin zeigen wir einheitliche Ansätze zu maschinenlesbaren Policy- und SLA-Definitionen. Anschließend entwickeln wir ein Service-Schema für den Multi-Cloud-Einsatz. Es folgen der Vorschlag für ein Broker-Design und passende Matching-Algorithmen.

\section{Die Limitierungen Kommerzieller CMPs}

Kommerzielle Anbieter wie \emph{RightScale} betonen den Self-Service-Charakter und potentielle Kosteneinsparungen durch ihrer Cloud-Management-Lösungen: Traditionell werden virtuelle Maschinen bei Bedarf von der internen IT bereitgestellt. Dabei entstehen auf der einen Seite Wartezeiten und auf der anderen erhöhter Arbeitsaufwand. 

Eine CMP kann diese Arbeiten automatisieren. Sie helfe laut \emph{Rightscale} die \emph{Schatten-IT} abzubauen -- die entsteht, wenn Mitarbeiter aufgrund langwieriger interner Prozesse zu Public-Cloud-Angeboten greifen -- mit allen negativen Folgen für Sicherheit und Vertraulichkeit. Die automatische Durchsetzung von Policys könnte sich also doppelt lohnen. Nebenbei liefert die CMP einen Preis- und Feature-Überblick der internen und öffentlichen Angebote. So hilft sie das optimale Angebot zu finden.

Nichtsdestotrotz sollte die CMP unabhängig entwickelt und betrieben werden. Cloud-Provider-eigene Lösungen werden daher nicht betrachtet. Weniger geeignet sind auch SaaS-Angebote: Hier entsteht eine neue Abhängigkeit und \emph{Single Point of Failure}. Der folgende Aufzählung zeigt die aktuell verbreitetsten CMP-Lösungen. Sie betrachtet besonders Bereitstellungsmodelle, Funktionsumfang und Offenheit der Schnittstellen.

\todo{(Zusatz) Tabelle}

%Automatisierung oder nur schöne Dashboards?

\begin{description}
	
	\item[Red Hat CloudForms\footnotemark]\footnotetext{\url{https://www.redhat.com/en/technologies/management/cloudforms/}}
	Kommerzielle Cloud Management Platform, Grundlage ist das Open Source-Projekt  ManageIQ\footnotemark\footnotetext{\url{https://manageiq.org/}}, das alle wichtigen IaaS-Provider unterstützt (AWS, Azure, GCP, OpenStack).	
	
	Besonderheit: ein umfangreiches -- optional grafisches -- Policy-Management. Eigene Regeln folgen dem Schema \emph{Bedingung/Ereignis-Aktion}.
	
	Alle Schnittstellen der CMP sind proprietär. Die Orchestrierung liest jedoch vorhandene Vorlagen aus \emph{AWS CloudFormation} and \emph{OpenStack Heat}.
	%http://manageiq.org/docs/reference/latest/doc-Policies_and_Profiles_Guide/miq/
	
	\item[Rightscale CMP\footnotemark]\footnotetext{\url{https://www.rightscale.com/}}
	Proprietäres \emph{Software-as-a-Service}-Angebot, unterstützt alle wichtigen IaaS-Provider, zusätzlich Plattformdienste, Docker-Container und Hypervisoren, besonders \emph{VMware vSphere}.
	
	Infrastruktur und Dienste werden als Vorlagen in einem eigenen Katalog bereitgestellt. Dabei erlaubt Rightscale auch heterogene Anwendungen über IaaS-, CaaS- und PaaS-Grenzen hinweg.
	
	Ein Dashboard zeigt Empfehlungen zur Kostenoptimierung, allerdings ohne SLAs einzubeziehen.
	
	\item[Scalr] 

	%	Whitepaper! 
\end{description}

% https://www.embotics.com/solutions-cloud-governance
% Nur Azure und Amazon. Cloud Governance: Die richtigen Meta-Tags zu Instanzen hinzufügen. Kostenoptimierungsvorschläge, aber nur die Wahl zwischen zwei Providern.




%DivvyCloud (Commercial) 
%
%Automation Bots to schedule downtime, terminate, or re-size instances and resources so you only pay for what you use 
%
%https://divvycloud.com/product/botfactory/for-cost/ 

%
%Commercial Tools 
%
%https://www.cloudyn.com/ 
%
%close-source 
%
%only cost monitoring and optimization 
%
%Also, Rightscale, Cloudhealth, CloudCheckr 


%Apache Scalr (complex, freemium) 

\section{Bisherige Forschungsarbeiten}

%policy-driven service placement optimization in federated clouds 


RESERVOIR (hard and soft requirements) 



OPTIMIS (Trust/Risk/Eco/Cost) 

canceled research project 

Dependable sociability = trust + risk + eco + cost 

introducing new multi-cloud HYBRID architectures with broker 

combination of federation and multi-cloud 

no geo-location 

non-disclosed algorithm 

no current external adapter available 

SP and IP, focus on IP benefit 





Contrail 

Federation 

Meta-data access (location, price..) 

although not usable in SLAs  :( 



Meryn 

SLA-driven PaaS system 

Optimize Cloud Provider costs 

Cloud Bursting 

Batch-application-centered (Hadoop) 

Single-use VMs 

Bidding 



InterCloud 

Pricing-aware 

Broker 

Federation 



Seaclouds EU project 

2013 

Cloud agnostic PaaS 

Discover/Planner/Deploy/Monitor/SLA-Services 

only monitors user-specific SLA 

Not used during planning/matchmaking 



Mist.io 

Open source 

Build on libcloud and cloudify 

Can monitor costs (commercial) 

No automatic scheduling 



An SLA-based Broker for Cloud Infrastructures 

Nicht nur Private Clouds sondern auch Personal Devices 

Modulare Architektur 

Fokus auf Föderation. Aber: Idee der forced integration of commercial cloud providers 

-> Mithilfe libcloud umsetzen 




Einige unterstützen über Adapter mehrere Bereitstellungsmodelle; je nachdem, ob dem Cloud-Provider die Mitgliedschaft in einer Föderation bewusst ist. Interne Adapter sprechen mit zusätzlichen, Cloud-internen Erweiterungen. Externe Adapter nutzen die öffentlichen APIs von Drittanbieter-Clouds.


Fazit: Die bisherigen Forschungsprojekte unterstützen keine Container-Technologien. Aus aktuellen \emph{DevOps}-Szenarien sind diese jedoch unverzichtbar.

Unterschiedliche Ausrichtung von Forschungsprojekten und Kommerziellen Anbietern Grozev: Forschungsprojekte fokussieren sich auf Föderationen. Dies entspricht auch dem vorherrschenden Cloud-Typ innerhalb der Projekte. SLAs werden ausschließlich von diesen ausgewertet und teilweise umgesetzt. Kommerzielle Projekte sind dagegen meist externe Broker, stellen Preisunterschiede und Kosten dar. Statt SLAs setzen sie nur einfachere Policys um.

Grozev: Universelle SLA-Spezifikation nötig

Code nur teilweise OpenSource, Matching nicht offengelegt. Nachvollziehbare Forschung sieht anders aus!

Proprietäre Schemata für Infrastruktur, Services und wenn verfügbar SLAs. 

APIs sind nicht einheitlich.

Kombination aus Pricing und Compliance bisher einmalig.






\section{Maschinenlesbare SLA- und Policy-Schemata}

Aufgreifen der Policy- und SLA-Anforderungen

Offen ist der Durchsetzungspunkt der Policys. Soll dies bereits in der CMP oder innerhalb von Plattformdiensten und Anwendungsprogrammen geschehen? Für Benutzer-Policys könnte beides der Fall sein: Die versendeten Anwendungsdaten enthalten Metainformationen zur erlaubten Verwendung. Schon die erste Kontaktstelle der Cloud, zum Beispiel ein Load-Balancer, wertet diese Informationen aus und entscheidet entsprechend für ein Routing, dass den Nutzeranforderungen entspricht. 

In der Anwendung selbst können diese Metainformationen auch genutzt werden, zum Beispiel zur gewünschten Verschlüsselung oder einer Frist zur automatischen Löschung. Ein besonders auf Performance fokussierter Ansatz ist [CPPL]. Bei System-Policys ist die Performance der Dekodierung jedoch nicht entscheidend: Sie findet nur bei Änderungen durch Administratoren statt, nicht bei jedem Datenpaket, das die Cloud erreicht.

Operational Level (Management, Computational)
Policy- und Event Definition Language
Anforderungen->Graph
% [2] T. Koch, C. Krell, and B. Krämer, "Policy definition language for automated management of distributed systems," In Proceedings of Second IEEE International Workshop on Systems Management, 1996, pp. 55-64.

%Per-Service Security SLa: A New Model for Security Management in Clouds


CPPA (SCICLOPS) End User - Object privileges on 
Tables 
Indices 
Procedures 
SQL operations, e.g.,  
Select 
Insert 
Update 
Delete 
Create… 

System-Privilegien: Anwendungsverteilung, also Sicht eines Administrator
Admin - System privileges 
System and Application Setup 
User Management
Monitoring/Logging
Backup

Bisherige Standards

CSA, OPTIMIS

einfach lesen für Mensch und Maschine, Versionsverwaltung, Infrastruktur als Code, erweiterbar

%%%%%%%%%%%%%%%%%%%%%%%%%%%%%%%%%%%%%%%%%%%%%%
%https://github.com/IntelLabsEurope/OCCI-SLAs%
%%%%%%%%%%%%%%%%%%%%%%%%%%%%%%%%%%%%%%%%%%%%%%

\todo{Verschieben, vor CMP-Marktübersicht}

\section{Einheitliche Infrastruktur- und Service-Definitionen}

Durch die Definition von SLAs und Policys steht nun fest, wo und wie ein Service bereitgestellt werden soll. Die konkrete technische Umsetzung ist bisher allerdings offen. Oberstes Ziel der CMP-Service-Definition ist Portabilität: Die gleiche Anwendung muss auf verschieden Hypervisoren, IaaS/CaaS- und Plattform-Angeboten ausgeführt werden. Nur so ergeben sich die Vorteile der Multi-Cloud-MP:

\begin{description}
	
	\item[Notfallplan] Bei Ausfall der gesamten Ausführungsplattform kann eine Migration zu einem anderen Anbieter erfolgen.
	
	\item[Abfangen von Lastspitzen] Zusätzliche Service-Instanzen auf externen Ressourcen bearbeiten bei Bedarf weitere Anfragen.
	
	\item[Lebenszyklus-Verwaltung] Typischerweise durchläuft eine neue Service-Version die Phasen Entwicklung, Test, Qualitätssicherung und Produktion. Hierfür existieren oft unterschiedliche Ausführungsumgebungen.
	
\end{description}

%Rightscale Video: https://www.rightscale.com/solutions/problems-we-solve/self-service-it

\noindent
Diese Vorteile zeigen sich jedoch nur, wenn sich die Anwendung auch für eine Cloud-Nutzung eignet. Je nach Entstehungsgeschichte können sich Software-Architektur und Migrations-Maßnahmen grundlegend unterscheiden:

\begin{description}
	
	\item[Klassisch] (Legacy-)Anwendung mit monolithischem Design
	
	$\Rightarrow$ Ausführungsumgebung portabel bereitstellen, z.\,B. als (Container-)Image
	
	\item[Web App] Mehrere skalierbare Komponenten
	
	$\Rightarrow$ Zusätzlich die Nutzbarkeit von PaaS-Komponenten prüfen
	
	\item[Cloud-Native] Abhängigkeiten zu proprietären Cloud-Services (interner Broker)
	
	$\Rightarrow$ Refactoring und Öffnung der Schnittstellen zur CMP
	
\end{description}

\noindent
Gerade bei klassischen, monolithischen Anwendungen ergeben sich Architektur-bedingt nicht alle Vorteile der Cloud-Nutzung; Eine erhöhte Portabilität ist gegeben --  Skalierbarkeit allerdings nicht. Die Entscheidung für Anpassung, Migration oder unverändertem Weiterbetrieb muss also je nach technischer Eignung und langfristiger Bedeutung für das Kerngeschäft abgewogen werden.

Ist die Entscheidung für eine Cloud-Migration gefallen, muss eine portable Infrastruktur geschaffen werden. Dieser Prozess besteht aus drei wesentlichen Schritten:

\begin{enumerate}
	
	\item Interpretation einer einheitlichen Service-Definition
	\\\emph{(Auswahl von Cloud-Ressourcen und passender Ausführungsform)}
	
	\item Bereitstellen von Infrastruktur-Ressourcen 
	\\\emph{(Virtuelle Maschine, Containerlaufzeitumgebung, Netzwerkspeicher etc.)}
	
	\item Übertragen der Anwendung in die Ausführungsumgebung
	\\\emph{(und initiale Konfiguration sowie Prüfung)}
	
\end{enumerate}

\noindent
SLAs und Policys werden hier noch nicht betrachtet. Der spätere Broker berücksichtigt sie vor allem während der Ressourcen-Auswahl in Schritt eins. Herausforderungen ergeben sich durch die Heterogenität von Cloud-Schnittstellen und Infrastruktur.

Durch die Marktdominanz von Amazon \emph{AWS} wurden zwischenzeitlich einige der proprietären Formate von Open-Source-Projekten übernommen: zum Beispiel \emph{Amazon Machine Images} (AMI, Cloud-optimierte Images) und \emph{CloudFormation} (Infrastruktur- und Service-Definitionen). Zielführend ist das jedoch nicht: Die Formate können sich jederzeit ändern, sind speziell auf Amazon-Angebote ausgerichtet und unterstützen im Gegenzug keine Eigenheiten anderer Infrastruktur. Dementsprechend haben sie sich nicht durchgesetzt:

\begin{description}
	
	\item[Ausführungsumgebungen] Innerhalb der Service-Ebenen stehen je nach Cloud-Provider unterschiedliche Ausführungsumgebungen zur Verfügung. Die zugehörige initiale Konfiguration einer neuen Instanz kann über interne Werkzeuge erfolgen oder Drittanbieter einbinden.
	
		Auf Hypervisor- und IaaS-Ebene sind \emph{Cloud Images} wie die von Ubuntu\footnote{\url{https://cloud-images.ubuntu.com/}} inoffizieller Standard. Im Gegensatz zu den Standard-Ausgaben sind sie speziell für den Einsatz in virtuellen Umgebungen vorbereitet. Sie integrieren \emph{Canonicals cloud-init\footnote{\url{https://cloudinit.readthedocs.io/}}}: Die initiale Konfiguration kann hierüber Provider-unabhängig per Konfigurationsdatei und/oder Skript übergeben werden. Optional bindet cloud-init über Plugins die Metadatendienste des Providers ein.
		
		Eine ähnliche Verbreitung auf CaaS-Ebene haben \emph{Docker}-Container. Besonderheiten sind die zentrale Container-Verwaltung, eine Vielzahl vorgefertigter Basis-Images und Konfigurationsmöglichkeiten per \emph{Dockerfile} und/oder Startparameter. Die Container sind so flexibel, dass sie lokal auf Entwicklerrechnern, im Continuous-Integration-Prozess und im Produktivbetrieb eingesetzt werden. Docker bildet außerdem die Grundlage für diverse PaaS-Projekte.
		
		\todo{Schaubild Rightscale}
		Während auf den bisherigen Service-Ebenen zumindest inoffizielle Standards existieren, ist die PaaS-Landschaft noch in großer Bewegung: Mit \emph{Open\-Shift} und \emph{Cloud\-Foundry} existieren mindestens zwei populäre Open-Source-Ansätze parallel zu den proprietären Angeboten der Public-Cloud-Provider.
	
	\item[Infrastruktur- und Service-Schemata] Eine Service-Definition soll die verschiedenen Komponenten einer Anwendung in Zusammenhang bringen und Abhängigkeiten festlegen. Auf CaaS-Ebene existieren hierfür unter vielen anderen \emph{Docker Compose} und \emph{Kubernetes}. PaaS-Services lassen sich über \emph{CloudFoundry} definieren. Alle drei Lösungen sind jedoch auf ihre Service-Ebene beschränkt -- daher können sie zwangsläufig nicht alle Anwendungsszenarien abbilden.
	
		Die Definition sollte erst einmal Provider-unabhängig erfolgen; So entsteht eine Topologie, die anschließend von einem Orchestrator oder Broker interpretiert und umgesetzt wird.
		
		Ein offenes, Provider- und Service-Ebenen-übergreifendes Schema ist die \emph{Topology and Orchestration Specification for Cloud Applications} (TOSCA\footnote{\url{http://docs.oasis-open.org/tosca/TOSCA-Simple-Profile-YAML/v1.1/TOSCA-Simple-Profile-YAML-v1.1.html}}). Sie unterstützt	weitere Details, unter anderem Variablen, Vererbung, Start- und Stopp-Aktionen sowie einfache Policys. Ein einfaches Service-Beispiel könnte wie folgt aussehen:
		
		\inputminted[]{yaml}{./src/TOSCA.sample.yml}\todo{Find a more interesting example}
		
		Die Kommunikation mit Service-Providern über die TOSCA-Referenz-Implementierung ist theoretisch möglich, die Plugins sind jedoch wenig verbreitet und als experimentell gekennzeichnet. Im Fall von OpenStack ist die Unterstützung bei \emph{Kilo} stehen geblieben. Eine externe Lösung ist also erfolgversprechender.
		
		Implementiert wird TOSCA außerdem von der Open-Source-CMP \emph{Cloudify\footnote{\url{https://cloudify.co/}}} und im EU-Forschungsprojekt \emph{SeaClouds}. Beide sind Teil der folgenden Übersichtskapitel.
		
		% Im Bereich PaaS überschneidet sich der Abschnitt Infrastruktur-Schemata architekturbedingt mit dem der Ausführungsumgebungen; diese wird meist einfach in einer Konfigurationsdatei deklariert und muss nicht selbst bereitgestellt werden.
	
	\item[Cloud-Provider-Schnittstellen] Die Kommunikation mit Cloud-Diensten ist nicht normiert. Jeder Provider implementiert eine andere (REST-)API und entwickelt eigene SDKs. An eine einheitliche Kommunikation ist nicht zu denken, denn austauschbare APIs widersprechen den Geschäftsinteressen der Public-Cloud-Provider.
	
	Mit dem \emph{Open Cloud Computing Interface (OCCI\footnote{\url{https://occi-wg.org/}})} existiert ein offener Standard.  
	
	Überschneidet sich teilweise mit TOSCA.
	
	SLAs direkt mit dem Cloud-Provider vereinbaren.
	
	Screenshot Tool!
	
	Experimenteller OpenStack Support
	
	
	\autoref{sec:bibliotheken}
	
		Offener Standard müsste von Cloud-Providern implementiert werden. Denkbar als Interface für eine eigene CMP. Pragmatisch ist der Einsatz einer Multi-Cloud-Bibliothek.
	
\end{description}

\noindent
Insgesamt ist der Support für offene Standards ist im Open-Source-IaaS-Projekt OpenStack am größten. Auf allen Ebenen der Service-Bereitstellung sind zumindest experimentelle Implementierung verfügbar: \emph{cloud-init} zur Imagekonfiguration, \emph{TOSCA} zur Service-Definition und \emph{OCCI} zur Kommunikation mit der Cloud.

Produkte die alles implementieren



Je nach Cloud unterscheiden sich die konkreten Inhalte dieser Bereitstellungsschritte.


\todo{Schaubild Template-Level: Skripte/Variablen, Multi-Cloud Image/Container, IaaS/CaaS, PaaS} 
% Rightscale?

Teillösungen greifen zu kurz: Greift zu kurz: Cloud Foundry PaaS (Theoretisch), aber nur PaaS, Migration von Monolithen unmöglich. Daher Kombination aus Hypervisor/IaaS/CaaS und PaaS nötig.
%https://blog.gruntwork.io/why-we-use-terraform-and-not-chef-puppet-ansible-saltstack-or-cloudformation-7989dad2865c 

Container sind nicht für jede Anwendung einsetzbar. Entsprechend müssen auch klassische virtuelle Maschinen unterstützt werden.

Maybe packer: Build Automated Machine Images for both.

%Juju https://jujucharms.com/canonical-kubernetes/ %
%Terraform % SaltStack -> Just Infrastructure
% Chef Puppet Ansible -> Configuration (Shouldn't need those)
%Kubernetes https://kubernetes.io/docs/tutorials/stateless-application/guestbook/ But that is just for deployment of Docker Images on already existing Kubernetes Clusters. Not Multi-CLoud from Scratch! %
%http://docs.getcloudify.org/4.2.0/blueprints/spec-policy-types/ %
%https://en.wikipedia.org/wiki/OASIS_TOSCA (Cloudify/SeaClouds) 

Cloudify implementiert den TOSCA-Standard. Mithilfe (grafischer) Werkzeuge lassen sich Cloud-Infrastrukturen und Services modellieren. Plugins erweitern Cloudify um alle wichtigen Provider auf IaaS-Ebene (AWS, Azure, GCP, OpenStack) als auch CaaS (Docker, Kubernetes). Konfiguration wie das Start-Skript kann direkt übergeben, oder über ein Tool wie Puppet weitergeleitet. Optionale(?) AGents of Hosts. Built-in Policies and rudimentäres Monitoring über Plugins.

Komplex, aber vielversprechend. Hierauf aufbauen (eigenen YAML-Entwurf erwähnen) und Brokering hinzufügen. Hier muss festgelegt erden, welcher Service-Teil auf welchem Provider mit welchem Instanz-Typen bereitgestellt werden soll. Dies soll automatisiert anhand von SLA/Policy und Preis entschieden werden. Unterstützt TOSCA deklarative Service-Definitionen?

Übernehmen?


%Read and Link Paper!



Einige weitere technische Herausforderungen löst der Multi-Cloud-Broker: Management von SSL-Zertifikaten, statischen und virtuellen IPs, sowie Load Balancing. Andere Migrationshürden bleiben: Verfügbarkeit von Betriebssystemen, Frameworks und Bibliotheken, ebenso wie die Vereinbarkeit mit vorhandenen Lizenzen.



\section{Modularer Architektur-Vorschlag}

Komponenten des Brokers.


In der CMP: Polling oder Notification?

Was löst eine Aktion aus?
- Monitoring der Services
- Änderung der Umgebung
- User-Aktion
- Ergebnis einer anderen Policy

\begin{figure}
	\centering
	\includegraphics[width=0.9\linewidth]{images/cycle}
	\caption{}
	\label{fig:Chicken1}
\end{figure}

Zyklus\autoref{fig:Chicken1}:

%\begin{description}
%	\item[Nummerierte Aufzählung]~\par
\begin{enumerate}
	
	\item Sammeln der Meta-Informationen alle Cloud-Provider
	\begin{enumerate}
		\item Kapazität (CPU, RAM, HDD, Network)
		\item Features (Verschlüsselung, CUDA, …)
		\item Geo-Lokation 
		\item Preis
	\end{enumerate}
	
	\item Sammeln der Laufzeitinformationen der PaaS/Anwendungen
	\begin{enumerate}
		\item Auslastung
		\item Fehler
		\item Ausfälle
	\end{enumerate}
	
	\item Sammeln der SLAs
	\begin{enumerate}
		\item Policy-Definitionen
		\item Policy-Konfiguration
		\item Placement-Algorithmen
	\end{enumerate}

	\item Neue Anwendung/Änderung eines SLA
	
	\item Optimierung
	\begin{enumerate}
		\item Feste Vorgaben (Geo, Backup)
		\item Weiche (Preis, Latenz, Verfügbarkeit)
	\end{enumerate}

	
	\item Ausführung
	\begin{enumerate}
		\item Netzwerkkonfiguration
		\item Allokation/De-Allokation von Ressourcen
		\item Deployment
		\item Migration
		\item Logging/Benachrichtigung
		\item Backup
	\end{enumerate}

\end{enumerate}
%\end{description} 

%\section{Schemata: Cloud-Angebote, SLAs, Services}



\section{Matching-Algorithmen}

OCCI-SLA

\todo{Schaubild, was wird wann gematcht}
%http://bl.ocks.org/gkatsaros/raw/a0a43032c4b50a246d4d/

Kostenoptimierung

Preisentwicklung? 

Migration je nach Tageszeit? 

Kosten der Datentransfers 

Subscription On-Demand/Monthly/Yearly 

Kompliziert durch undurchsichtige Staffelpreise
% https://www.rightscale.com/blog/cloud-cost-analysis/aws-vs-azure-vs-google-cloud-pricing-compute-instances

%https://www.rightscale.com/blog/cloud-cost-analysis/comparing-cloud-instance-pricing-aws-vs-azure-vs-google-vs-ibm

%
%Cost Calculators 
%
%http://go.appscale.com/cloud-cost-calculator-help 
%
%https://github.com/ifosch/accloudtant 
%
%https://awstcocalculator.com/# 
%
%


%
%Angebot der Cloud Provider (SLA)
%Anforderungen
%
%Spezifizierung
%Schnittstellen
%
%Algorithmen
	\include{content/3_loesung}
	\chapter{Entwurf und Implementierung}
\label{cha:implementierung}

\todo{Kapitel-Einleitung}



\section{Modularer Architektur-Vorschlag}

%Komponenten des Brokers.
%
%
%In der CMP: Polling oder Notification?
%
%Was löst eine Aktion aus?
%- Monitoring der Services
%- Änderung der Umgebung
%- User-Aktion
%- Ergebnis einer anderen Policy

\begin{figure}
	\centering
	\includegraphics[width=0.9\linewidth]{images/cycle}
	\caption{}
	\label{fig:cycle}
\end{figure}

%Zyklus\autoref{fig:cycle}:

%\begin{description}
%	\item[Nummerierte Aufzählung]~\par
\begin{enumerate}
	
	\item Sammeln der Meta-Informationen alle Cloud-Provider
	\begin{enumerate}
		\item Kapazität (CPU, RAM, HDD, Network)
		\item Features (Verschlüsselung, CUDA, …)
		\item Geo-Lokation 
		\item Preis
	\end{enumerate}
	
	\item Sammeln der Laufzeitinformationen der PaaS/Anwendungen
	\begin{enumerate}
		\item Auslastung
		\item Fehler
		\item Ausfälle
	\end{enumerate}
	
	\item Sammeln der SLAs
	\begin{enumerate}
		\item Policy-Definitionen
		\item Policy-Konfiguration
		\item Placement-Algorithmen
	\end{enumerate}
	
	\item Neue Anwendung/Änderung eines SLA
	
	\item Optimierung
	\begin{enumerate}
		\item Feste Vorgaben (Geo, Backup)
		\item Weiche (Preis, Latenz, Verfügbarkeit)
	\end{enumerate}
	
	
	\item Ausführung
	\begin{enumerate}
		\item Netzwerkkonfiguration
		\item Allokation/De-Allokation von Ressourcen
		\item Deployment
		\item Migration
		\item Logging/Benachrichtigung
		\item Backup
	\end{enumerate}
	
\end{enumerate}

\section{Brokering}


%https://de.wikipedia.org/wiki/Constraintprogrammierung
%https://de.wikipedia.org/wiki/Scheduling

%entailing multiple constraint satisfaction (MCS)
%
%\todo{Schaubild, was wird wann gematcht}
%% Pseudocode des Algorithmus, wie in Meryn
%
%Kostenoptimierung
%
%Preisentwicklung? 
%
%Migration je nach Tageszeit? 
%
%Kosten der Datentransfers 
%
%Subscription On-Demand/Monthly/Yearly 
%
%Kompliziert durch undurchsichtige Staffelpreise
% https://www.rightscale.com/blog/cloud-cost-analysis/aws-vs-azure-vs-google-cloud-pricing-compute-instances

%https://www.rightscale.com/blog/cloud-cost-analysis/comparing-cloud-instance-pricing-aws-vs-azure-vs-google-vs-ibm

%
%Cost Calculators 
%
%http://go.appscale.com/cloud-cost-calculator-help 
%
%https://github.com/ifosch/accloudtant 
%
%https://awstcocalculator.com/# 
%

\section{Testumgebung: OpenStack \& Hyrise-R}

Hyrise\footnote{\url{https://hpi.de/plattner/projects/hyrise.html}} ist eine In-Memory-Forschungsdatenbank der Fachgruppe \emph{Enterprise Platform and Integration Concepts (EPIC)} am Hasso-Plattner-Institut \cite{grund:2010:hyrise}. Die Datenbank teilt sich einige Eigenschaften mit \emph{SAP HANA}\footnote{\url{https://www.sap.com/products/hana.html}}: Ein \emph{Delta Store}, spaltenorientierte Speicherung, Wörterbuchkodierung und weitere Komprimierungstechniken sowie den \emph{Insert-Only}-Ansatz und Partitionierung. Herausragend ist die OLAP-Performance, enthalten sind aber auch Optimierungen für OLTP-Aufgaben.

Hyrise-R ist eine Erweiterung des Basisprojektes um Replikation \cite{schwalb:2015:hyrise-r}. Es folgt dabei dem \emph{Scale-Out}-Ansatz: Alle schreibenden Operationen werden auf einem einzigen \emph{Master-Node} durchgeführt. Dessen Datensatz wird in weniger als einer Sekunde (\emph{lazy}) mit beliebig vielen \emph{Replica-Nodes} abgeglichen. Diese Spiegelungen bearbeiten alle reinen Leseanfragen und machen den Verbund so skalierbar, siehe \autoref{fig:hyrise-r}. Nach dem \emph{CAP-Theorem} sind Verfügbarkeit und Partitionstoleranz hier also wichtiger als Konsistenz. 

	\begin{figure}[ht]
	\centering
	\def\svgwidth{\textwidth}
	\includesvg{images/hyrise-r}
	\caption{Verteilte \emph{Hyrise-R}-Architektur mit getrennter Verarbeitung von Lese- und Schreibanfragen. Der Master-Knoten dient als \emph{Single Source of Truth}. Zur Leistungssteigerung übernehmen Spiegelserver die Beantwortung der meisten Leseanfragen. Kleinere Inkonsistenzen werden dabei in Kauf genommen. Aus \cite{ssiclops:d42:experiments-measurements}.}	
	\label{fig:hyrise-r}
\end{figure}

Durch die verteilte Architektur ist Hyrise-R ein potenzieller Kandidat als Testanwendung innerhalb der Multi-Cloud-Umgebung. Einige \emph{SSICLOPS}-Teilprojekte untersuchten bereits Zuverlässigkeit, Performance, Datensicherheit und Vertraulichkeit in einer privaten OpenStack-Föderation \cite{ssiclops:d23:security-extensions, ssiclops:d42:experiments-measurements, bastian:2017:openstack-policies}. \todo{Diagramm:Hyrise-R on SSICLOPS}

Im Rahmen dieser Arbeiten sind einige Infrastrukturteile als Code veröffentlicht: So existiert zum Beispiel eine Docker-Teststellung mit grafischem Cluster-Manager, um die Performance bei verschiedenen Replikationsstufen zu prüfen. Diese Infrastruktur wurde in mehreren Studienarbeiten weiter angepasst, um Hyrise-R-KVM-Images in OpenStack bereitzustellen \cite{eschrig:2016:ssiclops-masterproject, maschler:2017:ssiclops-masterproject}. Möglicherweise können Teile dieser Arbeiten weiterentwickelt werden.


\section{Multi-Cloud-Bibliotheken}
\label{sec:bibliotheken}
\todo{Kleines Architektur Diagramm}

Ziel ist die Implementierung eines externen Broker-Services oder die Aufwertung einer verteilten Anwendung für den automatischen Betrieb in mehreren Clouds. Da unabhängige Cloud-Provider keine einheitlichen APIs anbieten, stellt dieses Kapitel verschiedene Bibliotheken vor, um möglichst viel der zusätzlichen Komplexität zu verbergen.

Ohne weitere Bibliotheken müsste für jede zu berücksichtigende Cloud das jeweilige SDK eingebunden werden. Auch Namensgebung, Architektur und Prozesse unterscheiden sich von Anbieter zu Anbieter. \todo{Mention Standard Cloud API OASIS TOSCA}

Durch den Einsatz einer Drittbibliothek ergibt sich allerdings eine potenzielle Schwachstelle. Falls diese fehlerhaft ist oder gar nicht weiter entwickelt wird, gefährdet dies das ganze Projekt. Historie und Zukunftschancen spielen bei der Auswahl eine zentrale Rolle. Im Optimalfall abstrahiert die Bibliothek Änderungen der Provider-SDKs. Ob und wie groß die Arbeitserleichterung ausfällt, prüft der Praxisteil.

Im Folgenden untersuchen wir die Eignung der populärsten Bibliotheken. Wichtigste Komponente ist dabei das Computing-Modul. Wünschenswert wäre auch Container-Unterstützung, um Images anbieterunabhängig bereitzustellen. Gestartete Anwendungskomponenten erfordern für die erste Erreichbarkeit oft Zugriff auf die DNS-Einstellungen der Cloud. Optional ist die Unterstützung von \emph{Content Delivery Networks}, Speicher- und Backup-Diensten.

% https://tex.stackexchange.com/questions/341592/hyphenating-text-inside-tabularx
\begin{table*}\centering
	\begin{minipage}{\textwidth}
	\caption{Übersicht freier Multi-Cloud-Bibliotheken. Mit $*$ gekennzeichnete Eigenschaften sind experimentell. Aufgeführt sind nur die populärsten Cloud-Provider, die Bibliotheken können darüber hinaus weitere unterstützen. Ob eine Bibliothek weitere Informationen, wie aktuelle Preisinformationen und den Standort des Rechenzentrums abrufen kann, zeigt die Spalte \emph{Cost\,/\,Geo}.}
	\ra{1.3}
	\begin{tabularx}{\textwidth}{>{\centering}XXXr} \toprule
		Projekt & Cloud-Provider & Cloud-Services & Cost\,/\,Geo\\ \midrule
		Apache Libcloud (Python)\footnotemark & AWS, Azure, OpenStack, GCP, Docker & Compute, Container, DNS, Load Balancer, Storage, Backup & $x$\,/\,$x$\\
		Apache jclouds (Java)\footnotemark & AWS, Azure, Open\-Stack$*$, GCP, Docker & Compute, Container, Load Balancer$*$, Storage & $x$\,/\,$x$\\
		PkgCloud (Node.js)\footnotemark & AWS, Azure, OpenStack& Compute, Load Balancer, Storage$*$, DNS$*$ & --\,/\,--\\
		Libretto (Go)\footnotemark & AWS, Azure, OpenStack, GCP & Compute & --\,/\,--\\
		Fog (Ruby)\footnotemark & AWS, OpenStack, GCP & Compute, DNS, Storage & $x*$\,/\,--\\
		\bottomrule
	\end{tabularx}
	\label{tab:bibliotheken}
	\vspace{150pt}
	\footnotetext[1]{\url{https://libcloud.apache.org/}}
	\footnotetext[2]{\url{https://jclouds.apache.org/}}
	\footnotetext[3]{\url{https://github.com/pkgcloud/pkgcloud/}}
	\footnotetext[4]{\url{https://github.com/apcera/libretto/}}
	\footnotetext[5]{\url{http://fog.io/}}
\end{minipage}  
\end{table*}

\autoref*{tab:bibliotheken} listet die untersuchten Bibliotheken mit unterstützten Cloud-Providern, Services und weiteren Features. Letzteres sind Zugriff auf Preisinformationen des Anbieters und Standortinformationen der Rechenzentren. Zusätzlich sollten die Projekte kontinuierlich weiterentwickelt werden, eine aktive Entwicklergemeinschaft besitzen und gut dokumentiert sein. Alle sind Open Source und unter einer freien Lizenz verfügbar.

\begin{description}
	
	\item[Apache jclouds] existiert schon seit 2009. Es unterstützt zumindest experimentell die wichtigsten Provider, aber nicht alle Services: DNS ist nicht vorhanden, Container-Unterstützung gibt es nur für Docker. Die Bibliothek ist gut getestet, dokumentiert, und mit zahlreichen Beispielen ausgestattet. Durch Java ist sie außerdem typsicher. 
	
	\emph{jclouds} ist außerdem Grundlage mehrerer Multi-Cloud-Projekte, z.\,B. von \emph{Apache brooklyn\footnote{\url{https://brooklyn.apache.org/}}}: Mithilfe von \emph{CAMP}-Plänen lassen sich Anwendungen über mehrere Clouds ausrollen.

	\item[Apache Libcloud] vereint viele Vorteile: Es unterstützt neben OpenStack, als Referenz für Private-Cloud-Installationen, alle großen und kleinen Cloud-Provider mit allen Kernservices. Besonders interessant ist der Container-Support für \emph{Docker}, \emph{Kubernetes}, \emph{Amazon ECS} und die \emph{Google Container Engine}. Entsprechend gepackte Anwendungen könnten in einer Vielzahl von Clouds ohne weitere Änderungen ausgeführt werden.

	\item[Fog] integriert die wichtigsten Anbieter und Services. Die Entwicklergemeinde rund um \emph{Fog} ist aktiv und die Bibliothek wird häufig eingesetzt. Besonders interessant sind die bereitgestellten Mocks, die Tests des neuen Services erleichtern sollen. Zumindest für OpenStack wird Metering unterstützt. Eine einheitliche Namensgebung der verschiedenen Cloud-Produkte existiert nicht.

	\item[Libretto] beschränkt sich ausdrücklich auf die Compute-Funktionalität mithilfe virtueller Maschinen. Das zugehörige Projekt ist aktiv, kommt aufgrund der fehlenden Funktionalität aber nicht infrage.

	\item[PkgCloud] ist die einzige bekannte \emph{Node.js}-Bibliothek. Funktionsumfang und einheitliche Namensgebung der Cloud-Services sind überzeugend; leider wird die Bibliothek seit dem Verkauf des federführenden Unternehmens nicht mehr aktiv gepflegt. Bereits eingereichte Pull Requests werden nicht bearbeitet. Damit scheidet \emph{PkgCloud} für das Projekt aus.

\end{description}

\noindent Vielversprechend war außerdem das \emph{Apache DeltaCloud}-Projekt: Aufbauend auf \emph{Ruby} stellt es nicht nur eine einheitliche API nach \emph{Cloud Infrastructure Management Interface}-Standard\footnote{\url{https://www.dmtf.org/standards/cloud}} für die Kernfunktionen der wichtigsten Cloud-Provider, sondern auch zusätzliche Client-Bibliotheken und Mock-Funktionen. Aufgrund des plötzlichen Rückzugs von \emph{Red Hat} erfolgt seit 2013 allerdings keine Weiterentwicklung mehr \cite{androu:2013:deltacloud-red-hat-end}. Dieses Beispiel zeigt die Wichtigkeit nicht-funktionaler Betrachtungen bei der Auswahl einer Bibliothek. Auch Apache-Top-Level-Projekte haben nicht unbedingt eine sichere, vorhersagbare Zukunft.

Darüber hinaus existieren spezialisierte Bibliotheken wie \emph{SimpleCloud}\footnote{\url{https://framework.zend.com/manual/1.11/de/zend.cloud.html}} auf \emph{PHP}-Basis, das allerdings eine feste Komponente im \emph{Zend Framework} ist. Auch gibt es neue Entwicklungen wie \emph{CloudBridge}\footnote{\url{https://github.com/gvlproject/cloudbridge}} auf \emph{Python}-Basis. Besonderheit hier: Die Abstraktionsschicht nutzt die nativen SDKs der Cloud-Provider. \emph{CloudBridge} ist leider noch in einem frühen Entwicklungsstadium und als experimentell gekennzeichnet.

\emph{Libcloud} fasst die verschiedenen Cloud-Angebote nicht nur in gemeinsamen Namensräumen zusammen, sondern normalisiert auch Leistungsklassen. Python erleichtert außerdem den Einstieg und fügt sich in viele \emph{Python}-basierte Systemautomatisierungen ein. Diese Multi-Cloud-Bibliothek wird also im weiteren Verlauf der Arbeit erprobt.

%https://brooklyn.apache.org/learnmore/theory.html
% Apache Brooklyn hat eine eigene YAML-Service-Description-Spezifikation, ähnlich zu CAMP, der Clou Application Management API. Die Integration von TOSCA ist geplant, und in einer anderen Arbeit bereit umgesetzt: 
%Trans-Cloud: CAMP/TOSCA-based Bidimensional Cross-Cloud
% Keine SLAs, sondern nur Trigger-Action-Policies.
% Nutzt intern jclouds zur Provider-Anbindung.

\section{OpenStack-Testbed}

Als Beispiel für eine Private-Cloud -- als Teil unseres Multi-Cloud-Setups -- soll OpenStack dienen. Es ist das populärste Open-Source-Projekt um eigene Infrastruktur als Service aufzubauen. Gesponsert wird es von Großunternehmen wie \emph{HPE}, \emph{IBM}, \emph{Canonical}, \emph{Red Hat} und anderen.

OpenStack setzt sich aus verschiedenen Teilprojekten zusammen, die jeweils einen Dienst entwickeln und bereitstellen. Ein Minimal-Setup besteht aus \emph{Nova} (Computing), \emph{Key\-stone} (Authentifizierung), \emph{Neutron} (Netzwerk) und \emph{Glance} (Images). Verbreitet sind außerdem \emph{Cinder} (Blockspeicher) und \emph{Horizon} (Dash\-board). Diese sollen auch in unserem Beispiel genutzt werden. Denkbar ist darüber hinaus die Integration eines Container-Dienstes. Der Zugriff auf die Infrastruktur erfolgt entweder über das Dashboard, Kommandozeilentools oder eine REST-API.

Grundsätzlich wäre auch der Aufbau einer OpenStack-Föderation wie in \emph{SSICLOPS} denkbar \cite{ssiclops:2015:d6.1-project-presentation}. Föderierte Cloud-Architekturen teilen sich zentrale Komponenten, in OpenStack mindestens den Authentifizierungsservice \emph{Keystone}. Je nach Föderationsvariante (\emph{Cells}, \emph{Regions}, \emph{Availability Zones} oder \emph{Host Aggregates}) sind auch Dienste wie Dashboard oder Speicher nur einmal vorhanden. Diese Architektur reduziert Fixkosten, erfordert allerdings spezielle Anpassungen innerhalb der Cloud. Auch gehen einige Vorteile wie Ausfallsicherheit und Unabhängigkeit der zentralen Dienste wieder verloren. Eine Kombination mit weiteren Cloud-Providern im Rahmen unseres Multi-Cloud-Setups ist denkbar, bleibt aufgrund der aufwendigen Einrichtung aber außen vor. Auch wäre der zusätzliche Erkenntnisgewinn vermutlich gering.\todo{SSICLOPS-OS-Architektur}

Selbst ein minimales OpenStack-Testsetup ist durch die diversen Dienste komplex. Denkbar wäre also auch die Nutzung von externen OpenStack-Angeboten. In diesem Projekt gibt es hierfür grundsätzlich drei mögliche Bereitstellungsmodelle: 

\begin{enumerate}
	\item Public Cloud
	\\\emph{Betrieb auf geteilter Cloud-Infrastruktur}
	
	\item Hosted Private Cloud
	\\\emph{Betrieb auf exklusiver Cloud-Infrastruktur}
	
	\item Lokale Testinstallation
	\\\emph{Betrieb auf eigener physischer oder virtueller Infrastruktur}
\end{enumerate}

\noindent Eine Liste öffentlicher OpenStack-Angebote findet sich auf der Projekthomepage\footnote{\url{https://www.openstack.org/marketplace/hosted-private-clouds/}}. Dort werden auch weitere Informationen wie Funktionsumfang und Zertifizierungen aufgeführt. 

Interessant ist zum Beispiel das Angebot der Deutschen Telekom \emph{Open Telekom Cloud\footnote{\url{https://cloud.telekom.de/en/infrastructure/open-telekom-cloud/}}}: eine Public Cloud auf OpenStack-Basis -- in Deutschland -- mit vollem Funktionsumfang und API-Zugriff. International bietet \emph{Rackspace} eine Hosted Private Cloud\footnote{\url{https://www.rackspace.com/openstack/}}. Beide eignen sich jedoch kaum, um kleine Experimente zu starten, sondern richten sich vor allem preislich an größere Organisationen und Unternehmen.

Kostenlos ist das Public-Cloud-Angebot \emph{TryStack}\footnote{\url{http://trystack.org/}}. Sponsoren wie \emph{Cisco}, \emph{NetApp}, \emph{Dell} und \emph{Red Hat} finanzieren das Projekt. Die Registrierung erfolgt über die Aufnahme in eine Facebook-Gruppe, anschließend soll hierüber auch der Zugang zur kostenlosen OpenStack-\emph{Liberty}-Instanz erfolgen. Während der gesamten Laufzeit dieser Arbeit war allerdings weder ein Login noch Kontakt zu den Organisatoren möglich.

Lokale OpenStack-Installationen sind aufwendig: Für jeden Dienst muss ein eigener physikalischer Rechner bereitstehen. Dementsprechend verweist die offizielle Dokumentation direkt auf die Vielzahl von OpenStack-Distributionen\footnote{\url{https://www.openstack.org/marketplace/distros/}}. Diese bieten fast immer einen vereinfachten Setup-Prozess und oft die Option statt physikalischen Rechnern virtuelle Maschinen oder Container zu nutzen. Wie auch bei den Hosted-Angeboten sind hier nicht alle Dienste verfügbar. In allen Paketen fehlt \emph{Zun}, der aktuelle Container-Service.

Speziell für lokale Test- und Entwicklungsumgebungen existiert \emph{DevStack}\footnote{\url{https://docs.openstack.org/devstack/latest/}}. Das offizielle OpenStack-Projekt installiert automatisiert die wichtigsten OpenStack-Dienste auf einer einzigen Maschine. Ausdrücklich unterstützt werden dabei auch VMs und \emph{LXC}-Container. Es soll daher als Erstes erprobt werden.


\section{DevStack virtualisiert inkl. Container-Support}

Während der Installation nimmt DevStack tief greifende Veränderungen am Hostsystem vor. Es müsste also auf einem separaten Server installiert werden. Dieser Abschnitt beschreibt den Versuch einer virtualisierten, reproduzierbaren DevStack-Testinstallation. Außerdem soll \emph{Zun} integriert werden. 

Ziel ist DevStack in einem Container auszuführen, genauso wie die darin gestarteten Compute-Nodes ebenso in einem -- nun verschachtelten -- Container bereitzustellen. Die Gründe hierfür sind zusammengefasst:

\begin{enumerate}
	\item Keine oder minimale Änderungen am Hostsystem
	\item Reproduzierbarer Testaufbau
	\item Schneller und rückstandsloser Reset
	\item Zustände (\emph{Snapshots}) speicherbar
	\item Schnelle Ausführung von Gastapplikationen
\end{enumerate}

\noindent Auch eine virtuelle Maschine löst die oben genannten Probleme. Theoretisch. Problematisch wird die Ausführungsgeschwindigkeit von Gastanwendungen in einem mit \emph{VirtualBox} virtualisierten OpenStack. Eine Lösung ist \emph{verschachteltes KVM}, das bereits in der Arbeit [1] erprobt wurde. Die Autoren empfehlen ihren Vorschlag bei bestehenden Erfahrungen mit \emph{libvirt}. Der damalige Versuchsaufbau stellt sich allerdings als instabil und nicht mehr reproduzierbar heraus.

\begin{figure}[ht]
	\centering
	\begin{subfigure}[b]{0.46\textwidth}
		\def\svgwidth{\linewidth}
		{\small
			\includesvg{images/devstack-bare-metal}}	
		\caption{Bare-Metal-Installation}
		\label{fig:sub:devstack-bare-metal}
	\end{subfigure}\hfill%
	\begin{subfigure}[b]{0.49\textwidth}
		\def\svgwidth{\linewidth}
		{\small
			\includesvg{images/devstack-vm}}	
		\caption{All-in-One-VM}
		\label{fig:sub:devstack-vm}
	\end{subfigure}\\[8pt]%
	\begin{subfigure}[b]{0.49\textwidth}
		\def\svgwidth{\linewidth}
		{\small
			\includesvg{images/devstack-docker}}	
		\caption{DevStack in Docker}
		\label{fig:sub:devstack-docker}
	\end{subfigure}
	\begin{subfigure}{0.45\textwidth}
	\end{subfigure}
	
	\caption{Verschiedene Installationsvarianten für eine OpenStack-Testinstallation mit DevStack auf einem einzelnen Host -- inklusive Unterstützung für Docker-Compute-Container und klassische VMs. Eine direkte Installation verändert unwiderruflich das gesamte Host-System \emph{(a)}. Eine VM benötigt mehr Ressourcen und kann die Geschwindigkeit der Gastanwendungen negativ beeinflussen \emph{(b)}. Die Installation in einem Container schafft Abstraktion und Reproduzierbarkeit ohne Geschwindigkeitskompromisse. Die Gastcontainer nutzen weiterhin den Kernel des Host-OS \emph{(c)}.}
	\label{fig:devstack}
\end{figure}

\emph{LXD}-Container könnten sich ebenfalls eignen. Im Gegensatz zu Docker führen sie mehrere Prozesse aus, erinnern also mehr an eine klassische virtuelle Maschine (ohne deren Overhead). Laut Entwickler \emph{Canonical} fokussiert sich \emph{LXD} speziell auf IaaS-Aufgaben\footnote{\url{https://www.ubuntu.com/containers/lxd}}. Ein LXD-DevStack-Setup birgt allerdings die gleichen Hürden\footnote{\url{https://docs.openstack.org/devstack/latest/guides/lxc.html}} wie ein Docker-Setup \cite{graber:2016:openstack-lxd}. Beachtenswert ist noch das OpenStack-Projekt \emph{Kolla}, das jeden OpenStack-Dienst in einem eigenen Docker-Container installiert\footnote{\url{https://cloudbase.it/openstack-kolla-hyper-v/}}.

Um Container innerhalb von OpenStack auszuführen, gibt es mehrere, teils konkurrierende Projekte. Alle lassen sich über Plugins in DevStack einbinden. Dies sind die wichtigsten \cite{singh:2017:containers-openstack}:

\begin{description}
	
	\item[Zun] Eigenständige OpenStack-API zum Starten und Verwalten von diversen Containertypen, inklusive \emph{Docker}\footnote{\url{https://wiki.openstack.org/wiki/Zun}}.
	
	\item[Nova Docker] Im Gegensatz zu \emph{Zun} erfolgt die Docker-Containerverwaltung über die bekannte Nova-API. Das Projekt wurde eingestellt\footnote{\url{https://wiki.openstack.org/wiki/Docker}}.
	
	\item[Nova LXD] Parallel zu \emph{Nova Docker} erfolgt der Zugriff über die Nova-API. Das Projekt wird von \emph{Canonical} aktiv vorangetrieben\footnote{\url{https://linuxcontainers.org/lxd/getting-started-openstack/}}. Weiterer Teil ist die Automatisierung via \emph{Juju}.
	
	\item[Magnum] Eine Self-Service-Lösung zur Orchestrierung auf Basis von \emph{Heat}. Stellt automatisiert Container Orchestration Engines (COEs) wie \emph{Docker Swarm} und \emph{Kubernetes} bereit\footnote{\url{https://wiki.openstack.org/wiki/Magnum}}.
	
\end{description}

\noindent DevStack in Docker wurde bereits vor einiger Zeit umgesetzt\footnote{\url{https://github.com/ewindisch/dockenstack}}. Da das Projekt nicht mehr gepflegt wird und auf das ebenfalls beendete \emph{Nova Docker} aufsetzt, erfolgt die Neuimplementierung mit folgenden Änderungen:

\begin{itemize}
	
	\item Ubuntu-LTS-Basis-Image 14.04 $\Rightarrow$ 17.10
	\item Mehrprozessunterstützung per \emph{systemd}\footnote{\url{https://docs.openstack.org/devstack/latest/systemd.html}}
	\item OpenStack-Version Kilo $\Rightarrow$ Pike
	\item libvirt/QEMU-Instanzen
	\item Nova Docker $\Rightarrow$ Zun
	\item Container-angepasste DevStack-Konfiguration
	\item Vollständige Netzwerkkonfiguration
	
\end{itemize}

\todo{Architektur-Diagramm}

\noindent Größte Hürde ist die Limitierung auf einen Prozess innerhalb eines Standard-Docker-Containers. Neuere DevStack-Versionen setzen auf \. Daher muss dies über die Umgebungsvariable \emph{ENV container docker} bekannt gemacht werden. Anschließend lässt sich \emph{systemd} über zwei weitere Workarounds starten\footnote{\url{https://github.com/moby/moby/issues/27202}}\footnote{\url{https://github.com/moby/moby/issues/9212}}.

\emph{Docker Build} bereitet das Image mit allen externen DevStack-Abhängigkeiten vor. Notwendige Dienste wie \emph{RabbitMQ} und \emph{MySQL} werden bereits im Voraus installiert. Das Container-Image führt beim Start nur noch die allerletzten Schritte des Setups aus. Ganz vorweg nehmen lässt sich das Setup nicht, weil während des Builds keine erweiterten Rechte vorliegen.

Nach erfolgreichem Start reicht das Kommando \emph{make run}, um per Zun einen \emph{Cirros}-Basis-Container\footnote{\url{https://docs.docker.com/samples/library/cirros/}} zu starten. Der Stand der gesamten OpenStack-Installation lässt sich per \emph{docker commit} oder experimentell per \emph{Docker-Snapshots}\footnote{\url{https://criu.org/Docker}} sichern.

Anpassbar sind im Skript OpenStack-Services und -Versionen, da DevStack direkt aus den Quellen installiert wird. So ändern sich allerdings selbst die Abhängigkeiten der als stabil gekennzeichneten Versionen. Das Prinzip Infrastruktur als Code geht hier nicht immer auf -- DevStack ist nicht zuverlässig reproduzierbar. \autoref{fig:devstack} vergleicht die Installationsvarianten.

Als \emph{Proof-of-Concept} ist die Integration von Docker, DevStack und Zun bisher einmalig. Der Code ist daher auf GitHub\footnote{\url{https://github.com/janmattfeld/DockStack}} veröffentlicht und zeigt einige \emph{Best Practices} und \emph{Lessons Learned} in Bezug auf die genannten Projekte.

Letztendlich greifen wir auf eine lokale \emph{Mirantis}-OpenStack-Installation aus dem \emph{SSICLOPS}-Projekt zurück. Die Infrastruktur ist virtuell und wird durch \emph{Fuel}\footnote{\url{https://www.mirantis.com/software/openstack/}} zuverlässiger wieder aufgebaut. Die Zun-Container-Dienste sind nicht enthalten; dafür aber alle anderen Kernfunktionen und APIs.

\section{Entwicklungsumgebung}

% Hyrise-R-OpenStack- und Docker-Images, wie ersxtellt?
% Capgemini Whitepaper Trend 2018, From Boring to sexy
% App->Cloud-Ntive S. 16

%% Hyrise-R 
%Bestehende verteilte Anwendungen für den EInsatz in der CLoud vorbereiten, STichwort CLoud Native.
%Ausgangslage? Git-Repository mit teilautmatisierten Shell-Skripten und Makefiles. Automatisierung der Build, Test und Produktions-Infrastruktur, Ubuntu 16.04 auf Bare-Metal, VM und (Docker)Container.
%Integrieren der bestehenden Tests in diese Umgebungen.
%Packen des Geamtpakets aus Ausführungsumgebung, Programm und (Test-)Daten. auch automatisiert. Die Konfiguration ist variabel. SIe wird schematisch in der App-Konfiguration vorgegeben und dann von der CMP während der initialen Bereitstellung oder späteren Re-Deployments angepasst und ausgeführt.

% Tatsächliche Broker Architektur
% Code-Eigenheiten
% Tests/KPIs/Validierung der Hypothese

\section{Softwarearchitektur}

%as in Grozev 42: Federated CLoud Management: There is a central repository of images. this is replicated to the specific iaas/caas providers on demand.
%
%Alle weiteren Managementprozesse sind für Clients transparent.


\section{Multi-Provider-Service-Schema}

% D2.1: Übersicht Policy-Sprachen: Performance und Speichergröße. Entgegengesetzte Interessen. Lesbarkeit über zweiteiliung: Einmal für Menschen, einmal auf Bit-ebene für Maschinen. SLA über Proxy

%D2.2: Policys auf allen Schichten

%Matthias Bastian: Policy in OpenStack.

...und SLAs.

Ziel: Portabilität.

Mensch-und maschinenlesbar

YAML als aktuellen Standard

%TOSCA komplex, aber vielversprechend. Hierauf aufbauen (eigenen YAML-Entwurf erwähnen) und Brokering hinzufügen. Hier muss festgelegt erden, welcher Service-Teil auf welchem Provider mit welchem Instanz-Typen bereitgestellt werden soll. Dies soll automatisiert anhand von SLA/Policy und Preis entschieden werden. Unterstützt TOSCA deklarative Service-Definitionen?
%
%
%TOSCA hat folgendes nur optional
%- YAML (als SimpleVersion)
%- Multi-Provider als Plugin (nicht gewartet)
%- 
\begin{listing}[ht]	
	\inputminted[]{yaml}{./src/provider.sample.yaml}
	\caption{Provider-Definition und Zugangsdaten. Der Broker liest alle eingetragenen Accounts automatisch ein und berücksichtigt sie bei der initialen Service-Bereitstellung sowie in Optimierungsläufen. Public-Clouds benötigen nur Zugangsdaten wie Benutzername und Passwort -- alle weiteren Informationen erfragt der Broker dynamisch zur Laufzeit vom Provider. In Private-Cloud-Umgebungen ist dies nicht immer möglich: Details zur Verfügbarkeit, geografische Lage und Kosten müssen manuell eingepflegt oder vom Monitoring festgestellt werden.}
	\label{listing:provider}
\end{listing}



Platzhalter werden mit Jinja während des Deployments gefüllt.

Ablage der Pläne als Dokumentation.

Broker durch Metainformationen (und Labels) der Instanzen theoretisch zustandslos -> Broker selbst ist nicht ausfallgefährdet.

Erklärung der Metainformationen (versionierbar), verschiedenen Parameter und Rollenbeschreibung.

Je Provider Angaben zu Image und Startkommando. Dies wird hier eingetragen, um vom Broker dynamisch mit aktuellen Variablen angepasst zu werden: IP, Port...

Abhängig vom Service Level: IaaS/CaaS. Auch PaaS ist so denkbar. (Angabe als Image, Interpretation durch den Broker)

Beispiel verlinken.
Kapitel: Legacy Services Hyrise

Image-Erstellung und Repository.

Eigene Befehle
- Cloud-Init (Standard)
- shell/bash (Docker)

Vordefinierte Policys z.B. zum Verhalten im Fehlerfall. Aber auch Zusatzinformationen: Wie ist die Zustandsprüfung auf Service-Ebene auszuführen. Wichtig für Monitoring der Verfügbarkeit (SLA).

Abhängigkeiten von Services und wie oft global vorhanden? Hier: global ein master, abhängig vom Dispatcher.

\begin{listing}[ht]	
	\inputminted[firstline=15]{yaml}{./src/hyrise-r.sample.yaml}
	\caption{Providerübergreifende Servicevorlage. Der Ausschnitt zeigt die Definition des zentralen \emph{Hyrise-R-Dispatcher}-Dienstes. Nicht zu sehen sind Metadaten und die übrigen Anwendungsbestandteile. Parameter werden zur Laufzeit vom Broker eingesetzt.}
	\label{listing:hyrise-r}
\end{listing}

Könnte auch zu einem konkreten CAMP-Plan umgewandelt werden. So wie TOSCAMP. Stattdessen nur ein Template Schema und Graph zur Laufzeit.
%https://brooklyn.apache.org/v/latest/blueprints/setting-locations.html


\section{Tests und Diskussion}

%Kosten: Rechenzeit und Bandbreite (außerhalb einer Cloud) also gegenläufiges Ziel zu Portabilität und Ausfallsicherheit, denn die geringsten Kosten fallen bei dem Betrieb in einer einzelnen Cloud eines Providers an.
%i) providers’ pricing models, (ii) application’s communication patterns and (iii) distribution of nodes over providers.
%https://www.google.de/url?sa=t&rct=j&q=&esrc=s&source=web&cd=1&ved=0ahUKEwju27vHs6DZAhUCWRQKHRv7BEcQFggrMAA&url=http%3A%2F%2Fwww.mikesmit.com%2Fwp-content%2Fpapercite-data%2Fpdf%2Fcloud2012.pdf&usg=AOvVaw3e6yhHYmhWBbIxtr7MqkuX

%verschiedene OpenStack-Versionen haben unterschiedliche Schnittstellen. Auch dies kann über die Middleware abgefangen werden. RefStack testet API, Rally testet performance und führt tempest-Tests aus.


Aufwand einer Multi-Cloud-Strategie

Umsetzung der Policys

Potential

Vorteile durch Multi-Cloud-Bibliotheken

Aufwand für ein Multi-Cloud-Testbed

\chapter{Zusammenfassung und Ausblick}

Einheitliche Standards zu Services, SLAs und Kommunikation.

Policys innerhalb von Instanzen 
%(allow SSH, check for security vulnerabilities)

Policys auf Datenebene

Ausbau zu einer produktiven CMP
Identity
Discovery
Monitoring
Dashboard

Trend: Serverless

Migrationshürden von Apps auf die CMP: OS Version, SSL-Zertifikate, statische und virtuelle IPs, Lizenzen, Load Balancing, Clustering, Bandbreite, Mandantenfähigkeit.

Failover-Handling nicht definiert. Im Moment: Bereitstellung eines Services mit der gleichen Adresse bei Ausfall. Weitere Arbeit auf Anwendungsebene (Hyrise-R) nötig. Oder Ausgliederung der Service-Discovery an ein externes Tool.

Überwachung der SLAs und Durchsetzen von Schadensersatz.

	%In der Schlussbetrachtung gibt es einen Rückblick, in dem Motivation und Thesen aus der Einleitung wieder aufgegriffen und abgerundet werden. Antworten auf in der Problemstellung aufgeworfene Fragen werden kurz und prägnant zusammengefasst. Ebenso sollte ein Ausblick auf offen gebliebene Fragen sowie auf interessante Fragestellungen, die sich aus der Arbeit ergeben, gegeben werden. Ein persönlich begründetes Fazit aus eigener Perspektive ist an dieser Stelle ebenfalls sinnvoll.
%

\chapter{Zusammenfassung und Ausblick}

\begin{description}
	
	\item[Notfallplan] Bei Ausfall der gesamten Ausführungsplattform kann eine Migration zu einem anderen Anbieter erfolgen.
	
	\item[Abfangen von Lastspitzen] Zusätzliche Service-Instanzen auf externen Ressourcen bearbeiten bei Bedarf weitere Anfragen.
	
	\item[Lebenszyklus-Verwaltung] Typischerweise durchläuft eine neue Service-Version die Phasen Entwicklung, Test, Qualitätssicherung und Produktion. Hierfür existieren oft unterschiedliche Ausführungsumgebungen.
	
\end{description}

%Rightscale Video: https://www.rightscale.com/solutions/problems-we-solve/self-service-it

Einheitliche Standards zu Services, SLAs und Kommunikation.

Policys innerhalb von Instanzen 
%(allow SSH, check for security vulnerabilities)

Policys auf Datenebene

Ausbau zu einer produktiven CMP
Identity
Discovery
Monitoring
Dashboard
CAMP-Pläne zum Austausch mit weiteren Cloud-Management-Plattformen.

Trend: Serverless

Migrationshürden von Apps auf die CMP: OS Version, SSL-Zertifikate, statische und virtuelle IPs, Lizenzen, Load Balancing, Clustering, Bandbreite, Mandantenfähigkeit.

Failover-Handling nicht definiert. Im Moment: Bereitstellung eines Services mit der gleichen Adresse bei Ausfall. Weitere Arbeit auf Anwendungsebene (Hyrise-R) nötig. Oder Ausgliederung der Service-Discovery an ein externes Tool.

Überwachung der SLAs und Durchsetzen von Schadensersatz.
	\chapter{Die Cloud -- Chancen und Herausforderungen}

Dieses Kapitel definiert die grundlegenden Charakteristika eines Cloud-Dienstes, die verschiedenen Service-Ebenen, Liefermodelle, Akteure und ihre Verantwortlichkeiten. Aus diesen Definitionen entwickeln sich zwei grundlegende Herausforderungen der Cloud-Nutzung:

\begin{enumerate}
	\item Datenschutz/Vertraulichkeit
	\item Portabilität von Daten und Anwendungen
\end{enumerate}

% Unternehmens-Nutzers auf der PaaS-Ebene
\noindent Je nach Cloud-Nutzung ergeben sich hierfür verschiedene Lösungsansätze, die im weiteren Verlauf gegeneinander abgegrenzt werden.

\section{Eigenschaften eines Cloud-Dienstes}

Unabhängig von Liefer- und Servicemodell zeichnet sich ein Cloud-Dienst durch bestimmte Merkmale aus. Konkret definieren übereinstimmend \emph{NIST Cloud Computing Reference Architecture}, \emph{IETF} und \emph{BSI-Grundschutzkatalog} folgende Eigenschaften:
% http://ws680.nist.gov/publication/get_pdf.cfm?pub_id=909505 
% https://www.bsi.bund.de/DE/Themen/ITGrundschutz/ITGrundschutzKataloge/Inhalt/_content/m/m04/m04446.html?nn=6604968
% https://www.ietf.org/archive/id/draft-khasnabish-cloud-reference-framework-08.txt

\begin{description}

	\item[On-demand Self-service] Ressourcen werden vom Cloud-Kunden selbstständig über ein Portal oder eine Web-Schnittstelle angefordert und anschließend automatisch provisioniert.
	
	\item[Breitbandzugriff] Die gemieteten Ressourcen werden über ein Netzwerk, typischerweise das Internet, bereitgestellt. Der Zugriff erfolgt über Standard-Schnittstellen wie HTTP; kann also von überall erfolgen und ist im Regelfall nicht auf bestimmte Geräte oder Software beschränkt.
	
	\item[Geteilte Infrastruktur] Die zugrundeliegenden physikalischen Ressourcen werden virtualisiert und flexibel unter mehreren Kunden aufgeteilt. Die vorhandene Hardware wird so möglichst optimal ausgelastet. Gleichzeitig ergeben sich hierdurch Datenschutzbedenken; die Daten einzelner Mandaten müssen streng getrennt sein.
	
	\item[Elastizität] Durch einen hohen Grad an Automatisierung werden Ressourcen zeitnah zur Verfügung gestellt. Lastspitzen können so ohne manuelle Eingriffe abgefangen werden.

	\item[Messbarkeit] Die Ressourcennutzung ist messbar und wird kontinuierlich überwacht. Abgerechnet wird zum Beispiel nach CPU-Zeit, Speicherkapazität oder Anzahl genutzter IP-Adressen.
	
\end{description}

\noindent Von klassischem IT-Outsourcing grenzt es sich durch Self-Service, Skalierbarkeit und geteilte Infrastruktur ab. Diese Eigenschaften bieten Kunden theoretisch Flexibilität und Kostenvorteile. In der Lösungssuche sollen diese positiven Aspekte möglichst erhalten bleiben.


\section{Service-Ebenen}

Je nach Auswahl des Cloud-Angebots lassen sich verschiedene Kernebenen unterscheiden. Diese bauen jeweils aufeinander auf und verbergen die Komplexität der darunterliegenden Ebenen. Je weiter sich die Abstraktion von der physikalischen Ebene entfernt, desto weniger lässt sich das Angebot durch den Kunden anpassen:

\begin{description}
	
	\item[Infrastructure as a Service (IaaS)] Die klassische Bereitstellung von Infrastruktur wie virtuellen Maschinen, Speicherplatz und Netzwerkdienstleistungen. Der Kunde ist hier selbst für die Administration zuständig, muss also Einrichtung und Wartung von Betriebssystemen, Treibern und Middleware selbst verantworten.
	
	\item[Platform as a Service (PaaS)] Hier übernimmt der Cloud Provider die Bereitstellung der zuvor genannten Bestandteile. Der Kunde betreibt auf dieser Ebene eine selbst erstellte Anwendungssoftware. Über Bibliotheken und Schnittstellen des Cloud Providers greift er auf Laufzeitumgebungen, Datenbanken und Entwicklungswerkzeuge zu.
	
	\item[Function as a Service (FaaS)] Auch als \emph{Serverless Computing} populär: Entgegen des Namens arbeiten auch hier noch Server, diese sind für den Kunden jedoch weitestgehend unsichtbar. Es stellt eine Evolution des PaaS-Modells dar und ist besser als FaaS beschrieben -- der Kunde lädt nur noch Quellcode in die Cloud. Dieser wird nun Ereignis-getrieben ausgeführt, skaliert und abgerechnet. Im Gegensatz zu vielen PaaS-Angeboten fallen im Ruhebetrieb keine weiteren Kosten an.
	%https://www.crisp-research.com/serverless-infrastructure-der-schmale-grat-zwischen-einfachheit-und-kontrollverlust/
	
	\item[Software as a Service (SaaS)] Eine bestehende Anwendungssoftware wird komplett vom Cloud Provider bezogen. Die Verantwortlichkeit des Kunden beschränkt sich meist auf kleinere Anpassungen, Nutzerverwaltung und das Einspielen eigener Daten.
	
\end{description}

\noindent
Besonders interessant für eigene Entwicklungen im Rahmen des aktuellen Forschungskontextes sind dabei die Ebenen IaaS und PaaS. Sie bieten genug Flexibilität um die Fragestellung mit folgenden Produkten zu erproben:

\begin{enumerate}
	\item OpenStack als zentraler Infrastrukturprovider
	\item Die verteilte Forschungsdatenbank Hyrise-R als Beispielanwendung
\end{enumerate}

\noindent
Cloud Provider bieten darüber hinaus weitere Hilfs- und Verwaltungsdienste. Diese betreffen vor allem Konfiguration, Provisionierung, Monitoring und Abrechnung. Hierzu zählen aber auch Sicherheit, Vertraulichkeit und Portabilität. Diese drei Querschnittsthemen sollen im Zusammenhang mit den Dienstebenen weiter untersucht werden.

Die NIST-Klassifizierung unterscheidet hier speziell weitere, teils externe, Akteure wie Cloud-Auditoren und -Carrier. Dieser Einteilung folgt die Arbeit nicht. Stattdessen konzentriert sie sich auf die direkte Beziehung zwischen Cloud-Kunden und -Provider. Beide haben Risiken und Verantwortlichkeiten, die im nächsten Abschnitt besprochen werden.

\section{Risiken}

Moderne IT-Infrastruktur ist hochkomplex. Allein hierdurch ergibt sich ein großes Potential für Bedrohungen. Im Cloud Computing steigt das Risiko durch gemietete und geteilte Infrastruktur weiter. Zusätzlich zu allgemeinen Risiken sollen mögliche Auswirkungen auf folgende Eigenschaften beschrieben werden:

\begin{enumerate}
	\item Verfügbarkeit
	\item Vertraulichkeit
	\item Integrität
	\item Portabilität
\end{enumerate}

\noindent
Abhängig vom Einsatzzweck des geplanten Cloud-Services resultieren die Fragen: Welche Informationen und Prozesse müssen geschützt werden, welche Bedrohungen sind zu erwarten? Damit ist nicht nur Sicherheit gemeint, sondern alle Risiken, die den Erfolg eines Cloud-Projektes oder einer Organisation darüber hinaus bedrohen.

Dabei hilft, den möglichen Schaden im Voraus zu berechnen. Zu verarbeitende Daten sollten kategorisiert werden; dem BSI folgend sind diese vier Abstufungen denkbar:

\begin{enumerate}
	\item Privat- , Geschäfts- und Dienstgeheimnisse gemäß §§\,203 und 353\,b StGB
	\item Personenbezogene Daten gemäß §\,3 Absatz\,1 BDSG
	\item Verschlusssachen
	\item Sonstige Daten (weder Kategorie 1, noch 2, noch 3)	
\end{enumerate}

\noindent
Verschlusssachen der Kategorie drei meint hier alle Daten, deren Verlust, Veränderung oder unrechtmäßige Herausgabe sich nachteilig auswirken könnte. Die Abgrenzung der letzten beiden Kategorien erscheint oft schwierig, muss jedoch für jeden betriebenen Cloud-Service abgewogen werden.

Diese Arbeit konzentriert sich auf die Risiken, die direkt von Cloud-Kunden und -Providern auf den Ebenen IaaS und PaaS beeinflusst werden können. So ist zum Beispiel die Sicherheit der Client-Geräte, von den aus auf die Cloud-Dienste zugegriffen wird entscheidend, aber nicht Teil dieser Betrachtung.

Aus der Datenkategorie ergeben sich Anforderungen an die Risikoanalyse. Je höher und je wahrscheinlicher ein potentieller Schaden, desto aufwendiger und teurer sollte die Absicherung ausfallen. Grundsätzlich lassen sich zwei Kategorien von Risikofaktoren unterscheiden (Quelle):
% https://www.researchgate.net/publication/308691801_Taxonomy_for_Identification_of_Security_Issues_in_Cloud_Computing_Environments
\begin{itemize}
	\item Menschlich
	\item Technisch
\end{itemize}

\noindent Menschliche Fehlhandlungen sind entweder absichtlich oder unabsichtlich. Dies können Fehlinterpretation von SLAs, Manipulationen, Angriffe durch Social Engineering oder schlicht Inkompetenz sein. Im Cloud Computing treten diese Risiken verstärkt auf, da die Infrastruktur von Dritten betreut wird. Auch Gesetzesänderungen zählen zu diesen Risiken. Viele lassen sich durch passende Standard-Prozesse, Notfallpläne, Rechtemanagement und Audits vermindern.

Technisch gilt Ähnliches: Klassische Risiken wie der Ausfall von Hardware wird vom Cloud-Provider vor dem Kunden verborgen. Speziell auf Cloud-Projekte bezogen eröffnen sich aber auch neue Angriffsflächen wie die Cloud-Plattform selbst. Die Virtualisierungsebene kann durch mangelhafte Mandanten-Trennung Datenlecks öffnen. Viele Cloud-Provider arbeiten mit proprietären Protokollen, die Portabilität ist also eingeschränkt. Umso herausfordernder wird ein Notfallplan, der den Ausfall des Anbieters abfangen soll.

All diese Risiken müssen durch SLAs und Policys abgebildet werden. Diese sollten maschinenlesbar sein, um automatisiert angewandt und überprüft zu werden. Detaillierte Leitfäden hierzu bieten BSI, CSA. Insgesamt hängt das Risiko stark vom Bereitstellungsmodell ab, also von Standort und Nutzerkreis der Infrastruktur.
% Sichere Nutzung von Cloud-Diensten | Der sichere Weg in die Cloud
% IT-Grundschutz: M 2.535 Erstellung einer Sicherheitsrichtlinie für die Cloud-Nutzung
% BSI: Testierung nach BSI Anforderungskatalog Cloud Computing C5
% ISO/IEC 27001:2013, Information securitymanagement [13]
%Cloud Security Alliance (CSA) Security, Test & Assurance Registry (STAR): STAR Self Assess- ment, STAR Certification, STAR Attestation, C-STAR Assessment

%\section{Policy und SLA}

\section{Bereitstellungsmodelle und Multi-Cloud-Architekturen}

Cloud-Angeboten können von öffentlichen Anbietern oder intern bereitgestellt werden. Weiter klassifizieren lassen sich die Angebote nach Nutzerkreis, mit dem die Infrastruktur geteilt wird und Anzahl der genutzten Clouds:

\begin{description}
	
	\item[Public Cloud] Alle Leistungen werden von einem öffentlichem Anbieter bezogen. Dies sind zum Beispiel Amazon, Microsoft und Google. Die Infrastruktur wird unter mehreren Kunden flexibel aufgeteilt.
	
	\item[Private Cloud] Eine eigen- oder fremd-betriebene Infrastruktur mit exklusivem Zugriff für einen Kunden. Wird die Private Cloud im eigenen Datencenter betrieben, erhalten Kunden größtmögliche Vertraulichkeit. Gleichzeitig müssen aber Überkapazitäten vorgehalten werden, wodurch der Kostenvorteil kleiner als bei Nutzung öffentlicher Angebote ausfallen kann.
	
	\item[Hybird Cloud] Heterogene Infrastruktur mit Bestandteilen in privaten und öffentlichen Cloud-Umgebungen. Die öffentliche Cloud übernimmt hier oft die Speicherung großer Datenmengen und das Abfangen von Lastspitzen.
	
	\item[Community Cloud] Ein Zusammenschluss von Unternehmen, Behörden oder Forschungseinrichtungen, die gemeinsam eine Cloud-Infrastruktur betreiben. Die beteiligten Cloud-Anbieter bilden eine freiwillige \emph{Föderation} der gemeinsamen Ressourcen.
	
	\item[Multi-Cloud] Eine Erweiterung der Hybrid Cloud: Die Leistungen werden nicht nur aus einer privaten und einer öffentlichen Cloud bezogen, sondern explizit aus mehreren. Wichtiger Unterschied zur Föderation: Die Multi-Cloud-Umgebung wird vom Cloud-Anwender initiiert und verwaltet. Beteiligte Cloud-Provider wirken nicht aktiv mit und sind sich ihrer Partizipation meist nicht einmal bewusst.
	
\end{description}

\noindent
Von allen beschriebenen Bereitstellungsmodellen ist die Multi-Cloud am flexibelsten. Je nach technischen und nicht-funktionalen Anforderungen können Bestandteile der Cloud-Anwendungen in einer möglichst passenden Umgebung ausgeführt werden. Der Aufwand ist in diesem Modell allerdings auch am höchsten, denn der Auswahlprozess für einen Cloud-Provider muss für alle Anbieter einzeln durchlaufen werden.

Hierbei werden die Rahmenbedingungen geprüft, zum Beispiel Kosten, Standort der Rechenzentren und anwendbare Gesetzesgrundlage. Hinzu kommen technische Herausforderungen, da die Portabilität von Daten und Anwendungen zwischen verschiedenen Cloud-Providern oft eingeschränkt ist.

Fazit: Cloud-Nutzungsprozess wie in BSI "Sicher Nutzung von Cloud-Diensten" S.26\todo{Vergleichs-grafik zum neuen Prozess}, aber nicht Einweg, sondern kontinuierlich: Während des Betriebs kontinuierlich an neue Rahmenbedingungen anpassen, automatisch erkennen und migrieren. Daher: Cloud Broker.

Projektübergreifende Nutzung des Brokers. 
%\section{Cloud-Broker}

% Föderierte Cloud-Architekuren: Auch eine geteilte Nutzung zentraler Komponenten ist denkbar. In OpenStack teilen sich die verschiedenen Multi-Site-Clouds mindestens den Authentifizierungs-Service Keystone. Je nach Varainte (Cells, Regions, Availability Zones oder Host Aggregates) sind auch Dienste wie die Admin-Oberfläche oder Speicher nur einmal vorhanden. Diese Architektur spart Overhead, erfordert allerdings spezielle Anpassungen innerhalb der Cloud. Auch gehen einige Vorteile wie Ausfallsicherheit und Unabhängigkeit der zentralen Dienste wieder verloren. Eine Kombination mit weiteren Cloud-Providern ist denkbar, bleibt hier auf Grund der aufwändigen Einrichtung aber außen vor.



%\section{Related Work}
% 
%Eigene Rolle: Consumer. AGBs und SLAs meist nicht verhandelbar. Geteilte Verantwortung: Das beste daraus machen!
%
%Service-Taxonomy / Cloud-Typen
%
%
%Risiken
%
%
%Anforderungen
%
%Verwaltung über automatisierte Tools zur Orchestrierung.
%Kleine Marktübersicht?
%
%
%WAS IST EIN BROKER IN DIESEM KONTEXT?
%- Nicht Nutzerdaten werden geroutet.
%- Bestandteile verteilter Anwendungen.
%- Begründen.
%
%Klassifizierung der Broker.
%Ngrozev % example
	\include{content/chapter2} % example
%	\include{content/chapter3} % example

	% ggf. Anhang
%	\appendix\include{content/appendix} % example

	% Bibliographie
%	\ifisbook\cleardoubleemptypage\fi
%	\phantomsection\addcontentsline{toc}{chapter}{\refname}
%	\printbibliography[category=cited]

	% Eigenständigkeitserklärung
	\ifisbook\pagestyle{plain}\cleardoubleemptypage\include{content/disclaimer}\fi

\end{document}